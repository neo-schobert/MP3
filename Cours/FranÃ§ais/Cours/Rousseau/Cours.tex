\documentclass[a4paper, 11pt, hidelinks]{article}
\usepackage{bookmark}
\usepackage[utf8]{inputenc} 
\usepackage[T1]{fontenc}
\usepackage{lmodern}
\usepackage{graphicx}
\usepackage[french]{babel}
\usepackage{geometry}
\usepackage{eucal}
\usepackage{caption}
\usepackage{float}
\usepackage{url}
\usepackage{amsmath}
\usepackage{amssymb}
\usepackage{color}
\usepackage{hyperref}
\usepackage{cancel}
\usepackage{romanbar}
\usepackage{titlesec}

\geometry{hmargin=2cm,vmargin=1.5cm}

\newcommand{\dstylesum}{\displaystyle\sum}

\newcommand{\dstyleprod}{\displaystyle\prod}

\newcommand{\prp}{\large \textbf{Proposition :} \large}

\newcommand{\tm}{\large \textbf{Théoreme :} \large}

\newcommand{\ex}{\textcolor{green}{Exemple :} }

\newcommand{\dm}{\textcolor{red}{\textbf{Démo :} } }

\newcommand{\de}{\large \textbf{Définition} \large }

\newcommand{\rmq}{\textbf{Remarque :} }

\newcommand{\bs}{\bigskip}

\newcommand{\voca}{\textcolor{blue}{\textbf{Vocabulaire} } }

\newcommand{\cit}{\large \textcolor{blue}{\textbf{Citation :}} \large }

\newcommand{\rb}[1]{\Romanbar{#1}}
\newcommand{\trinom}[3]{\begin{pmatrix}
    #1 \\
    #2 \\
    #3
\end{pmatrix}}

\newcommand{\quadrinom}[4]{\begin{pmatrix}
    #1 \\
    #2 \\
    #3 \\
    #4 \\
\end{pmatrix}}

\newcommand{\pentanom}[5]{\begin{pmatrix}
    #1 \\
    #2 \\
    #3 \\
    #4 \\
    #5
\end{pmatrix}}

\newcommand{\hexanom}[6]{\begin{pmatrix}
    #1 \\
    #2 \\
    #3 \\
    #4 \\
    #5 \\
    #6 
\end{pmatrix}}

\newcommand{\serie}[2]{\displaystyle\sum_{#1 =0}^{+\infty} #2_{#1} }

\newcommand{\tend}{\underset{n \to + \infty}{\longrightarrow} }

\newcommand{\Lra}{\Leftrightarrow}

\newcommand{\lra}{\leftrightarrow}

\newcommand{\Ra}{\Rightarrow}

\newcommand{\ra}{\rightarrow}

\newcommand{\la}{\leftarrow}

\newcommand{\La}{\Leftarrow}

\newcommand{\dsum}[2]{\displaystyle\sum_{#1}^{#2} }

\newcommand{\dint}[2]{\displaystyle\int_{#1}^{#2} }

\newcommand{\ntend}{\underset{n \to + \infty}{\not \longrightarrow} }

\newenvironment{lmatrix}{$ \left|\begin{array}{l} }{\end{array}\right.$}


\setcounter{secnumdepth}{4}

\titleformat{\paragraph}
{\normalfont\normalsize\bfseries}{\theparagraph}{1em}{}
\titlespacing*{\paragraph}
{0pt}{3.25ex plus 1ex minus .2ex}{1.5ex plus .2ex}



\begin{document}




\title{L'enfance dans Rousseau, l'Emile}
\author{Francine Burlaud}

\maketitle

\tableofcontents


\newpage




\section{Questions}



\begin{enumerate}
    \item Pourquoi faut-il un gouverneur pour l'enfant ?
    \item Qu'elle education doit donner l'homme ?
    \item Comment savoir ce qui est bon et mauvais pour l'enfant
    \item Comment doit etre le maitre ?
    \item Que signifie bon chez rousseau.
    \item Que peut-on dire sur les exigences sur le gouverneur.
    \item Que peut-on dire de plus sur le gouverneur (sentiments)
    \item Que peut-on dire sur le gouverneur par rapport à l'argent ?
    \item Caractérisation du gouverneur.
    \item L'enfant doit-il être libre ou se croire libre ?
    \item Que dit Rousseau par rapport au raisonnement.
    \item Que ne faut-il pas enseigner aux enfants.
\end{enumerate}


\section{Réponses}


\begin{enumerate}
    \item suppléer à ses faiblesses (manque de force...)
    \item Il doit donner le strict necessaire dans l'education et pas d'artifice. Supprimer l'artifice laisser la nature. 
    \item En observant l'enfant.
    \item il doit etre exemplaire.
    \item vertueux.
    \item C'est mission impossible. Il doit être un saint.
    \item Il doit être aimé. (pas seulement talentueux)
    \item Ca doit pas être une tepu. (p105)
    \item jeune, bien éduqué, parle bien, capable de modération, de sang froid, ne cherche pas à briller à travers son élève.
    \item Il ne faut pas s'empresser de les faire raisonner
    \item La géographie, les langues, la chronologie...
\end{enumerate}


\cit "Soyez vertueux et bons, que vos exemples se gravent dans la mémoire de vos élèves, en attendant qu'ils puissent
entrer dans leurs c\oe urs."

\cit "Qu'il croie toujours être le maître et que ce soit toujours vous qui le soyez". (\rb{2}, p265)


\cit "Il y a un excès de rigueur et un excès d'indulgence, tout deux également à éviter." (\rb{2}, 166)


\cit "A chaque instruction précoce qu'on veut faire entrer, on plante un vice au fond de leur c\oe ur." (\rb{2})


\cit "Ils [les enfants] raisonnement très bien dans tout ce qu'ils connaissent et qui se rapporte à leur intérêt présent et sensible." (\rb{2})


\cit "Accorder aux enfants plus de liberté véritable et moins d'empire, leur laisser plus faire par eux même et moins exiger d'autrui." (\rb{1})


\de Liberté: fait de ne pas dépendre d'autrui.






\end{document}