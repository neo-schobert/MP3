\documentclass[a4paper, 11pt, hidelinks]{article}
\usepackage{bookmark}
\usepackage[utf8]{inputenc} 
\usepackage[T1]{fontenc}
\usepackage{lmodern}
\usepackage{graphicx}
\usepackage[french]{babel}
\usepackage{geometry}
\usepackage{eucal}
\usepackage{caption}
\usepackage{float}
\usepackage{url}
\usepackage{amsmath}
\usepackage{amssymb}
\usepackage{color}
\usepackage{hyperref}
\usepackage{cancel}



\geometry{hmargin=2cm,vmargin=1.5cm}
\newcommand{\prp}{\large \textbf{Proposition :} \large}

\newcommand{\tm}{\large \textbf{Théoreme :} \large}

\newcommand{\ex}{\textcolor{green}{Exemple :} }

\newcommand{\dm}{\textcolor{red}{\textbf{Démo :} } }

\newcommand{\de}{\large \textbf{Définition} \large }

\newcommand{\rmq}{\textbf{Remarque :} }

\newcommand{\bs}{\bigskip}

\newcommand{\voca}{\textcolor{blue}{\textbf{Vocabulaire} } }

\newcommand{\trinom}[3]{\begin{pmatrix}
    #1 \\
    #2 \\
    #3
\end{pmatrix}}

\newcommand{\quadrinom}[4]{\begin{pmatrix}
    #1 \\
    #2 \\
    #3 \\
    #4 \\
\end{pmatrix}}

\newcommand{\pentanom}[5]{\begin{pmatrix}
    #1 \\
    #2 \\
    #3 \\
    #4 \\
    #5
\end{pmatrix}}

\newcommand{\hexanom}[6]{\begin{pmatrix}
    #1 \\
    #2 \\
    #3 \\
    #4 \\
    #5 \\
    #6 
\end{pmatrix}}

\newcommand{\serie}[2]{\displaystyle\sum_{#1 =0}^{+\infty} #2_{#1} }

\newcommand{\tend}{\underset{n \to + \infty}{\longrightarrow} }

\newcommand{\Lra}{\Leftrightarrow}

\newcommand{\lra}{\leftrightarrow}

\newcommand{\Ra}{\Rightarrow}

\newcommand{\ra}{\rightarrow}

\newcommand{\la}{\leftarrow}

\newcommand{\La}{\Leftarrow}

\newcommand{\dsum}[2]{\displaystyle\sum_{#1}^{#2} }

\newcommand{\dint}[2]{\displaystyle\int_{#1}^{#2} }

\newcommand{\ntend}{\underset{n \to + \infty}{\not \longrightarrow} }

\newenvironment{lmatrix}{$ \left|\begin{array}{l} }{\end{array}\right.$}

\begin{document}




\title{Langages}
\author{Schobert Néo}

\maketitle

\tableofcontents


\newpage 

\large Dates DS : \large

$20$ octobre

$15$ décembre

$9$ février

$23$ mars

$30$ mars



\section{Origine de l'informatique et but}
But de l'informatique ?


Créée en 1936 (Article d'Alan Turing)


Turing cherche à répondre aux question naïve :

\begin{itemize}
\item Peut-on automatiser des démonstrations automatiques.
\item Une démonstrations est-elle équivalente à un algorithme.
\item Qu'est-ce qu'un calcul.
\end{itemize}

En cherchant à répondre, Turing conçoit des outils (machine de Turing) qui est devenu l'ordinateur.


La machine de Turing est une machine abstraite de mémoire infini.


La réponse : non, tout n'est pas résoluble par algorithme.


Informatique : science de l'information / pas de lien avec l'ordinateur en soi.


\bs

En pratique, en informatique, on cherche à résoudre à l'aide d'algorithmes, des problèmes.

Mais qu'est-ce qu'un problème ?

\voca Ils sont de plusieurs types:

\begin{itemize}
    \item d'évaluation
    \item d'optimisation
    \item d'approximation
    \item d'énumération
    \item de comptage
    \item de décision
\end{itemize}


Généralement, les problèmes d'informatique sont de type décision:

Une entrée (propriétés initiales)

Question (où on peut répondre par oui ou par non)

\bs 

Problème : La représentation des objets. En information, on utilise les flottants.
\bs

\voca L'instance est un codage d'objet abstrait.

\bs
\voca Un codage est une succession de symbole qui forme ce qu'on appelle un mot.

\bs
\voca L'ensemble des symboles forme un alphabet.

\bs
\voca L'ensemble des mots forme un langage formel.

\bs
Un modèle de calcul est une notion importante: il donne la manière de calculer informatiquement parlant.
\bs

\voca C'est un objet mathématique formé d'ensembles, de fonctions, de relations, qui abstrait
les propriétés en définissant une notion d'opération atomique.


\section{Alphabet et mot}


\de Un alphabet $\Sigma$ est un ensemble fini non vide de symboles $a_i$,
$i=1,...,k$ appelés lettres ou caractères.

$\Sigma= \{a_1,a_2,...,a_k\}$

\bs

\ex $\Sigma=\{a,b,...,z,let,rec,for,=,+,*,...\}$

\bs

\de Un mot $m$ est une suite finie de symboles sur un alphabet $\Sigma$.

Un mot n'est rien d'autre qu'un $n-uplet$.

\bs

\ex La suite $(a_{i1},a_{i2},...,a_{in}) \in \Sigma^n$ est un mot.

On omet les parenthèses traditionnellement : $m=a_{i1} a_{i2}... a_{in}$

\bs 

La longueur d'un mot $m$ est alors $|m|= n$

Le mot vide noté $\epsilon$ est l'unique mot de longueur $0$

$|mm'|=|m|+|m'|$.


L'ensemble des mots de longueur $n$ sur $\Sigma$ est noté $\Sigma^n$

L'ensemble de tous les mots sur $\Sigma$ est noté $\Sigma^+$ avec :

$\Sigma^+= \displaystyle\bigcup_{i=1}^{+\infty} \Sigma^i$

\bs

\de Soit $\Sigma$ un alphabet. La concaténation est une loi de composition interne, 

notée $.$, sur $\Sigma^*$.

Si $m=a_{i1} a_{i2}... a_{in}$ et $m'=a'_{i1} a'_{i2}... a'_{in}$ sont des mots construits sur 
$\Sigma$, la concaténation $m\dot m'$ (ou $mm'$) de $m$ et $m'$ est définie par :

$m\dot m' =a_{i1} a_{i2}... a_{in} a'_{i1} a'_{i2}... a'_{in}$

$\Sigma$ muni de cette loi forme un monoïde.

\bs
\de (préfixe)

\bs
\de (suffixe)

\bs
\de (facteur)

\section{Langages formel}

\de Un langage sur un alphabet $\Sigma$ (ou langage de $\Sigma^*$) est un ensemble de mots de $\Sigma^*$

Un langage est un sous-ensemble de $\Sigma^*$.

\bs
\de (union de langages)

\bs
\de (intersection de langages)

\bs
\de (concaténation de langages)

\bs
\de La fermeture étoilée (ou itération) du langage $L$, noté $L^*$ est le langage :

$L^*=\displaystyle\bigcup_{k \in \mathbb{N}^*} L^k$














































\end{document}