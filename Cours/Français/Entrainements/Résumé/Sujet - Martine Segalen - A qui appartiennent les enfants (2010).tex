\documentclass[a4paper, 11pt, hidelinks]{article}
\usepackage{bookmark}
\usepackage[utf8]{inputenc} 
\usepackage[T1]{fontenc}
\usepackage{lmodern}
\usepackage{graphicx}
\usepackage[french]{babel}
\usepackage{geometry}
\usepackage{eucal}
\usepackage{caption}
\usepackage{float}
\usepackage{url}
\usepackage{amsmath}
\usepackage{amssymb}
\usepackage{color}
\usepackage{hyperref}
\usepackage{cancel}
\usepackage{romanbar}
\usepackage{titlesec}

\geometry{hmargin=2cm,vmargin=1.5cm}

\newcommand{\dstylesum}{\displaystyle\sum}

\newcommand{\dstyleprod}{\displaystyle\prod}

\newcommand{\prp}{\large \textbf{Proposition :} \large}

\newcommand{\tm}{\large \textbf{Théoreme :} \large}

\newcommand{\ex}{\textcolor{green}{Exemple :} }

\newcommand{\dm}{\textcolor{red}{\textbf{Démo :} } }

\newcommand{\de}{\large \textbf{Définition} \large }

\newcommand{\rmq}{\textbf{Remarque :} }

\newcommand{\bs}{\bigskip}

\newcommand{\voca}{\textcolor{blue}{\textbf{Vocabulaire} } }

\newcommand{\cit}{\large \textcolor{blue}{\textbf{Citation :}} \large }

\newcommand{\rb}[1]{\Romanbar{#1}}
\newcommand{\trinom}[3]{\begin{pmatrix}
    #1 \\
    #2 \\
    #3
\end{pmatrix}}

\newcommand{\quadrinom}[4]{\begin{pmatrix}
    #1 \\
    #2 \\
    #3 \\
    #4 \\
\end{pmatrix}}

\newcommand{\pentanom}[5]{\begin{pmatrix}
    #1 \\
    #2 \\
    #3 \\
    #4 \\
    #5
\end{pmatrix}}

\newcommand{\hexanom}[6]{\begin{pmatrix}
    #1 \\
    #2 \\
    #3 \\
    #4 \\
    #5 \\
    #6 
\end{pmatrix}}

\newcommand{\serie}[2]{\displaystyle\sum_{#1 =0}^{+\infty} #2_{#1} }

\newcommand{\tend}{\underset{n \to + \infty}{\longrightarrow} }

\newcommand{\Lra}{\Leftrightarrow}

\newcommand{\lra}{\leftrightarrow}

\newcommand{\Ra}{\Rightarrow}

\newcommand{\ra}{\rightarrow}

\newcommand{\la}{\leftarrow}

\newcommand{\La}{\Leftarrow}

\newcommand{\dsum}[2]{\displaystyle\sum_{#1}^{#2} }

\newcommand{\dint}[2]{\displaystyle\int_{#1}^{#2} }

\newcommand{\ntend}{\underset{n \to + \infty}{\not \longrightarrow} }

\newenvironment{lmatrix}{$ \left|\begin{array}{l} }{\end{array}\right.$}


\setcounter{secnumdepth}{4}

\titleformat{\paragraph}
{\normalfont\normalsize\bfseries}{\theparagraph}{1em}{}
\titlespacing*{\paragraph}
{0pt}{3.25ex plus 1ex minus .2ex}{1.5ex plus .2ex}



\begin{document}




\title{Résumé de texte. Sujet - Martine Segalen - A qui appartiennent les enfants (2010)}
\author{Schobert Néo}

\maketitle

\tableofcontents


\newpage



\textbf{Corrigé:}


On a fait par paragraphe: (De 1 à 12)



\begin{enumerate}
    \item Le monde offre nombre d'exemples d'enfances malmenées , généralement 
    causées par une défaillance de l'Etat.
    \item En revanche, selon les Occidentaux, à l'idéologie marquées par la religion chrétienne, le rôle 
    parental comprends conception, alimentation, éducation et socialisation.
    \item (Le 3 était peut intéréssant)
    \item 
    \item Quand la notion de parenté se dissout dans une foule de pairs, l'enfant n'a pas de famille assignée, multipliant les liens 
    à des structures familiales variées.
    \item Généralement, le transfert parental se fait vers un foyer plus aisé. Ces procédés comblent les défaillances des Etats.
    \item (en général, on garde les dates) En europe, depuis le \rb{19}e siècle, l'Etat doit combler les 
    lacunes de l'éducation domestique dont la violence infantile est \begin{lmatrix}
        la \; cons\'equence \\
        le \; reflet 
    \end{lmatrix}
    \item Aujourd'hui, les familles délaissées cèdent leurs enfants au crime et au délit. La misère sociale des parents
    peut-elle excuser ces dérives ?
    \item L'entourage des enfants est devenu surprotecteur,
    \item et leur responsables toujours plus impliqués légalement.
    \item Toute société a sa vision de l'enfance; en Occident, elle est sacralisée.
    \item Idéalement, l'Etat et tous les partenaires éducatifs collaborent pour favoriser l'égalité des chances.
\end{enumerate}






\end{document}