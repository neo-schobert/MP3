\documentclass[a4paper, 11pt, hidelinks]{article}
\usepackage{bookmark}
\usepackage[utf8]{inputenc} 
\usepackage[T1]{fontenc}
\usepackage{lmodern}
\usepackage{graphicx}
\usepackage[french]{babel}
\usepackage{geometry}
\usepackage{eucal}
\usepackage{caption}
\usepackage{float}
\usepackage{url}
\usepackage{amsmath}
\usepackage{amssymb}
\usepackage{color}
\usepackage{hyperref}
\usepackage{cancel}
\usepackage{tikz}
\usepackage{mathrsfs}  
\usepackage{esvect}
\usepackage[standard]{ntheorem}
\usepackage{romanbar}
\usepackage{titlesec}



\geometry{hmargin=2cm,vmargin=1.5cm}

\tikzset{
  treenode/.style = {shape=rectangle, rounded corners,
                     draw, align=center,
                     top color=white, bottom color=blue!5},
  root/.style     = {treenode, font=\Large, bottom color=red!10},
  env/.style      = {treenode, font=\ttfamily\normalsize},
  dummy/.style    = {circle,draw}
}

\newcommand{\prp}{\large \textbf{Proposition :} \large}

\newcommand{\tm}{\large \textbf{Théoreme :} \large}

\newcommand{\ex}{\textcolor{green}{Exemple :} }

\newcommand{\dm}{\textcolor{red}{\textbf{Démo :} } }

\newcommand{\de}{\large \textbf{Définition} \large }

\newcommand{\rmq}{\textbf{Remarque :} }

\newcommand{\bs}{\bigskip}

\newcommand{\voca}{\textcolor{blue}{\textbf{Vocabulaire} } }

\newcommand{\lem}{\textcolor{red}{\textbf{Lemme :} } }

\newcommand{\rb}[1]{\Romanbar{#1}}


\newcommand{\trinom}[3]{\begin{pmatrix}
    #1 \\
    #2 \\
    #3
\end{pmatrix}}

\newcommand{\quadrinom}[4]{\begin{pmatrix}
    #1 \\
    #2 \\
    #3 \\
    #4 \\
\end{pmatrix}}

\newcommand{\pentanom}[5]{\begin{pmatrix}
    #1 \\
    #2 \\
    #3 \\
    #4 \\
    #5
\end{pmatrix}}

\newcommand{\hexanom}[6]{\begin{pmatrix}
    #1 \\
    #2 \\
    #3 \\
    #4 \\
    #5 \\
    #6 
\end{pmatrix}}

\newcommand{\serie}[2]{\displaystyle\sum_{#1 =0}^{+\infty} #2_{#1} }

\newcommand{\tend}{\underset{n \to + \infty}{\longrightarrow} }

\newcommand{\Lra}{\Leftrightarrow}

\newcommand{\lra}{\leftrightarrow}

\newcommand{\Ra}{\Rightarrow}

\newcommand{\ra}{\rightarrow}

\newcommand{\la}{\leftarrow}

\newcommand{\La}{\Leftarrow}

\newcommand{\dsum}[2]{\displaystyle\sum_{#1}^{#2} }

\newcommand{\dint}[2]{\displaystyle\int_{#1}^{#2} }

\newcommand{\ntend}{\underset{n \to + \infty}{\not \longrightarrow} }

\newenvironment{lmatrix}{$ \left|\begin{array}{l} }{\end{array}\right.$}

\newcommand{\img}[4]{\begin{figure}[!ht]
    \centering
    \includegraphics[scale=#1 ]{#2}
    \caption{#3}
    \label{#4}
    \end{figure} }    
\begin{document}

\newcommand{\grad}[1]{\vv{grad}#1}


\title{Livret de citations}
\author{Schobert Néo}

\maketitle

\tableofcontents


\newpage














\section{Nature / Condition humaine}



\subsection{Rousseau}


\begin{enumerate}
    \item Tout est bien sortant des mains de l'Auteur des choses, tout dégénère entre les mains de l'homme. (Livre \rb{1})
    \item Nos têtes seraient mal de la façon de l'Auteur de notre être: il nous les faut façonner au-dehors par les sages-femmes et au-dedans par les philosophes. (Livre \rb{1})
    \item Nous sommes atteints par des maux très sains; la nature elle-même, qui nous a fabriqués dans la rectitude, nous vient en aide dans la mesure où nous voulons nous améliorer/ (Livre \rb{2})
    \item Vis selon la nature, sois patient, et chasse les médecins; tu n'éviteras pas la mort, mais tu ne la sentiras qu'une fois, tandis qu'ils la portent chaque jour dans ton imagination troublée, et que leur art mensonger, au lieu de prolonger tes jours, t'en ôte la jouissance. (Livre \rb{2})
    \item Posons pour maxime incontestable que les premiers mouvements de la nature sont toujours droits: il n'y a point de perversité originelle dans le c\oe ur humain; il ne s'y trouve pas un seul vice dont on ne puisse dire comment et par où il y est entré. (Livre \rb{2})
    \item le mensonge de fait n'est pas naturel aux enfants; mais c'est la loi de l'obéissance qui produit la nécessité de mentir, parce que l'obéissance étant pénible, on s'en dispense en secret le plus qu'on peut, et que l'intérêt présent d'éviter le châtiment ou le reproche l'emporte sur l'intérêt éloigné d'exposer la vérité. (Livre \rb{2}) 
    \item
\end{enumerate}



\subsection{Soyinka}


\begin{enumerate}
    \item 
\end{enumerate}




\subsection{Andersen}


\begin{enumerate}
    \item 
\end{enumerate}










\section{Etat d'enfance / Force / faiblesse / Intelligence}



\subsection{Rousseau}


\begin{enumerate}
    \item La race humaine eût périt si l'homme n'eût commencé par être enfant. (Livre \rb{1})
    \item Thétis, pour rendre son fils invulnérable, le plongea, dit la fable, dans l'eau du Styx. (Livre \rb{1})
    \item Un corps débile affaiblit l'âme. (Livre \rb{1})
    \item La nature a fait les enfants pour être aimés et secourus; mais les a-t-elle faits pour être obéis et craints ? (Livre \rb{2})
    \item Avec leur force se développe la connaissance qui les met en état de la diriger. C'est à ce second degré que commence proprement la vie de l'individu; c'est alors qu'il prend conscience de lui-même. La mémoire étend le sentiment de l'identité sur tous les moments de son existence; il devient véritablement un, le même, et par conséquent déjà capable de bonheur ou de misère. Il importe donc de commencer à le considérer ici comme un être moral. (Livre \rb{2})
    \item Voulez-vous donc cultiver l'intelligence de votre élève; cultivez les forces qu'elle doit gouverner. Exercez continuellement son corps; rendez-le robuste et sain, pour le rendre sage et raisonnable; qu'il travaille, qu'il agisse, qu'il coure, qu'il crie, qu'il soit toujours en mouvement; qu'il soit homme par la vigueur, et bientôt il le sera par la raison. (Livre \rb{2})
\end{enumerate}



\subsection{Soyinka}


\begin{enumerate}
    \item 
\end{enumerate}




\subsection{Andersen}


\begin{enumerate}
    \item 
\end{enumerate}












































\section{Clivage enfants / adultes}



\subsection{Rousseau}


\begin{enumerate}
    \item La race humaine eût périt si l'homme n'eût commencé par être enfant. (Livre \rb{1})
    \item Il faut considérer l'homme dans l'homme et l'enfant dans l'enfant. (Livre \rb{2})
\end{enumerate}



\subsection{Soyinka}


\begin{enumerate}
    \item 
\end{enumerate}




\subsection{Andersen}


\begin{enumerate}
    \item 
\end{enumerate}
































\section{Société / Patrie}



\subsection{Rousseau}


\begin{enumerate}
    \item Tout patriote est dur aux étrangers; ils ne sont qu'hommes, il ne sont rien à ses yeux. (Livre \rb{1})
    \item Ces deux mots patrie et citoyens doivent être effacés des langues modernes. (Livre \rb{1})
    \item Les hommes ne sont point faits pour être entassés en fourmilières, mais épars sur la terre qu'ils doivent cultiver. Plus ils se rassemblent, plus ils se corrompent. Les infirmités du corps ainsi que les vices de l'âme, sont l'infaillible effet de ce concours trop nombreux. L'homme est de tous les animaux celui qui peut le moins vivre en troupeaux. Des hommes entassés comme des moutons périraient tous en très peu de temps. L'haleine de l'homme est mortelle à ses semblables: cela n'est pas moins vrai au propre qu'au figuré. (Livre \rb{1})
    \item quand les pauvres ont bien voulu qu'il y eût des riches, les riches ont promis de nourrir tous ceux qui n'auraient de quoi vivre ni par leur bien ni par leur travail. (Livre \rb{2})
    \item Un père, quand il engendre et nourrit les enfants, ne fait de cela que le tiers de sa tâche. Il doit des hommes à son espèce, il doit à la société des hommes sociables; il doit des citoyens à l'Etat. Tout homme qui peut payer cette triple dette et ne le ffait pas est coupable. (Libre \rb{1})
    \item 
\end{enumerate}



\subsection{Soyinka}


\begin{enumerate}
    \item 
\end{enumerate}




\subsection{Andersen}


\begin{enumerate}
    \item 
\end{enumerate}




























\section{Temps / Nostalgie}



\subsection{Rousseau}


\begin{enumerate}
    \item Que de voix vont s'élever contre moi ! J'entends de loin les clameurs de cette fausse sagesse qui nous jette incessamment hors de nous, qui compte toujours le présent pour rien, et, poursuivant sans relâche un avenir qui fuit à mesure qu'on avance, à force de nous transporter où nous ne sommes pas, nous transporte où nous ne serons jamais. (Livre \rb{2})
    \item La prévoyance ! La prévoyance qui nous porte sans cesse au-delà de nous, et souvent nous place où nous n'arriverons point, voilà la véritable source de toutes nos misères. Quelle manie a un être aussi passager que l'homme de regarder toujours au loin dans un avenir qui vient si rarement, et de négliger le présent dont il est sûr ! (Livre \rb{2})
\end{enumerate}



\subsection{Soyinka}


\begin{enumerate}
    \item 
\end{enumerate}




\subsection{Andersen}


\begin{enumerate}
    \item 
\end{enumerate}




































\section{Imagination / Curiosité}



\subsection{Rousseau}


\begin{enumerate}
    \item Vis selon la nature, sois patient, et chasse les médecins; tu n'éviteras pas la mort, mais tu ne la sentiras qu'une fois, tandis qu'ils la portent chaque jour dans ton imagination troublée, et que leur art mensonger, au lieu de prolonger tes jours, t'en ôte la jouissance. (Livre \rb{2})
    \item Le monde réel a ses bornes, le monde imaginaire est infini; ne pouvant élargir l'un, rétrécissons l'autre; car c'est de leur seule différence que naissent toutes les peines qui nous rendent vraiment malheureux. (Livre \rb{2})
\end{enumerate}



\subsection{Soyinka}


\begin{enumerate}
    \item 
\end{enumerate}




\subsection{Andersen}


\begin{enumerate}
    \item 
\end{enumerate}





























\section{Sens}



\subsection{Rousseau}


\begin{enumerate}
    \item Tant qu'il ne se connaît que par son être physique, il doit s'étudier par ses rapports avec les choses: c'est l'emploi de son enfance; (Livre \rb{2})
    \item les enfants, n'étant pas capables de jugement, n'ont point de véritable mémoire. Ils retiennent des sons, des figures, des sensations, rarement des idées, plus rarement leurs liaisons. (Livre \rb{2})
    \item Il veut tout toucher, tout manier: ne vous opposez point à cette inquiétude; elle lui suggère un apprentissage très nécessaire. C'est ainsi qu'il apprend à sentir la chaleur, le froit, la dureté, la molesse, la pesanteur, la légèreté des corps, à juger de leur grandeur, de leur figure, et de toutes leurs qualités sensibles, en regardant, palpant, écoutant, surtout en comparant la vue au toucher, en estimant à l'oeil la sensation qu'ils feraient sous ses doigts. (Livre \rb{1})
\end{enumerate}



\subsection{Soyinka}


\begin{enumerate}
    \item 
\end{enumerate}




\subsection{Andersen}


\begin{enumerate}
    \item 
\end{enumerate}
































\section{Savoir et savoir-vivre}



\subsection{Rousseau}


\begin{enumerate}
    \item Faute de savoir se guérir, que l'enfant sache être malade: cet art supplée à l'autre, et souvent réussit beaucoup mieux; c'est l'art de la nature. \rb{1}
    \item l'assujettissement de l'homme à la douleur, aux maux de son espèce, aux accidents, aux périls de la vie, enfin à la mort; plus on le familiarisera avec toutes ces idées, plus on le guérira de l'importune sensibilité qui ajoute au mal l'impatience de l'endurer; plus on l'apprivoisera avec les souffrances qui peuvent l'atteindre, plus on leur ôtera, comme eût dit Montaigne, la pointure de l'étrangeté; et plus aussi l'on rendra son âme invulnérable et dure. (Livre \rb{2})
    \item Les approches mêmes de la mort n'étant point la mort, à peine la sentira-t-il comme telle; il ne mourra pas, pour ainsi dire, il sera vivant ou mort, rien de plus. (Livre \rb{2})
\end{enumerate}



\subsection{Soyinka}


\begin{enumerate}
    \item 
\end{enumerate}




\subsection{Andersen}


\begin{enumerate}
    \item 
\end{enumerate}














































\section{Tyrannie et Obéissance de l'enfance}



\subsection{Rousseau}


\begin{enumerate}
    \item Il faut qu'il donne des ordres ou qu'il en reçoive. Ainsi ses premières idées sont celles d'empire et de servitude. (Livre \rb{1})
    \item La nature a fait les enfants pour être aimés et secourus; mais les a-t-elle faits pour être obéis et craints ? (Livre \rb{2})
    \item Si l'on ne doit rien exiger des enfants par Obéissance, il s'ensuit qu'ils ne peuvent rien apprendre dont ils ne sentent l'avantage actuel et présent, soit d'agrément, soit d'utilité; autrement quel motif les porterait à l'apprendre ? (Livre \rb{2})
    \item D'abord il voudra la canne que vous tenez; bientôt il voudra votre montre; ensuite il voudra l'oiseau qui vole; il voudra l'étoile qu'il briller; il voudra tout ce qu'il verra: à moins d'être Dieu, comment le contenterez-vous ? (Livre \rb{2})
    \item Accoutumés à voir tout fléchir devant eux, quelle surprise, en entrant dans le monde, de sentir que tout leur résiste, et de se trouver écrasés du poids de cet univers qu'ils pensaient mouvoir à leur gré! (Livre \rb{2})
\end{enumerate}



\subsection{Soyinka}


\begin{enumerate}
    \item 
\end{enumerate}




\subsection{Andersen}


\begin{enumerate}
    \item 
\end{enumerate}


































\section{Emancipation / Liberté}



\subsection{Rousseau}


\begin{enumerate}
    \item Vivre est le métier que je lui veux apprendre. En sortant de mes mains, il ne sera j'en conviens, ni magistrat, ni soldat, ni prêtre; il sera premièrement homme. (Livre \rb{1})
    \item Vivre, ce n'est pas respirer, c'est agir. (Livre \rb{1})
    \item Il n'y a qu'une seule chose à enseigner aux enfants: c'est celle des devoirs de l'homme. (Livre \rb{1})
    \item Il eût fallu veiller sans cesse sur un enfant en liberté; mais, quand il est bien lié, on le jette dans un coin sans s'embarrasser de ses cris. (Livre \rb{69})
    \item Le seul qui fait sa volonté est celui qui n'a pas besoin, pour la faire, de mettre les bras d'un autre au bout des siens; d'où il suit que le premier de tous les biens n'est pas l'autorité, mais la liberté. L'homme vraiment libre ne veut que ce qu'il peut, et fait ce qu'il lui plaît. Voilà ma maxime fondamentale. Il ne s'agit que de l'appliquer à l'enfance, et toutes les règels de l'éducation vont en découler. (Livre \rb{2})  
    \item Il faut qu'ils sautent, qu'il courent, qu'ils crient, quand ils en ont envie. Tous leurs mouvements sont des besoins de leur constitution, qui cherche à se fortifier. (Livre \rb{2})
\end{enumerate}



\subsection{Soyinka}


\begin{enumerate}
    \item 
\end{enumerate}




\subsection{Andersen}


\begin{enumerate}
    \item 
\end{enumerate}



































\section{Caprice}



\subsection{Rousseau}


\begin{enumerate}
    \item 
\end{enumerate}



\subsection{Soyinka}


\begin{enumerate}
    \item 
\end{enumerate}




\subsection{Andersen}


\begin{enumerate}
    \item 
\end{enumerate}






































































\section{Perversion}



\subsection{Rousseau}


\begin{enumerate}
    \item Tout est bien sortant des mains de l'Auteur des choses, tout dégénère entre les mains de l'homme. (Livre \rb{1})
    \item La nature veut que les enfants soient enfants avant que d'être hommes. Si nous voulons pervertir cet ordre, nous produirons des fruits précoses, qui n'auront ni maturité ni saveur, et ne tarderont pas à se corrompre; nous aurons de jeunes docteurs et de vieux enfants. (Livre \rb{2})
    \item Les hommes ne sont point faits pour être entassés en fourmilières, mais épars sur la terre qu'ils doivent cultiver. Plus ils se rassemblent, plus ils se corrompent. Les infirmités du corps ainsi que les vices de l'âme, sont l'infaillible effet de ce concours trop nombreux. L'homme est de tous les animaux celui qui peut le moins vivre en troupeaux. Des hommes entassés comme des moutons périraient tous en très peu de temps. L'haleine de l'homme est mortelle à ses semblables: cela n'est pas moins vrai au propre qu'au figuré. (Livre \rb{1})

\end{enumerate}



\subsection{Soyinka}


\begin{enumerate}
    \item 
\end{enumerate}




\subsection{Andersen}


\begin{enumerate}
    \item 
\end{enumerate}














































\section{Raisonnement / Logique / Bien et Mal}



\subsection{Rousseau}


\begin{enumerate}
    \item Dialogue Maître / élève (Livre \rb{2}) : \begin{itemize}
                                    \item LE MAÎTRE Il ne faut pas faire cela.
                                    \item L'ENFANT Et pourquoi ne faut-il pas faire cela ?
                                    \item LE MAÎTRE Parce que c'est mal fait.
                                    \item L'ENFANT Mal fait ! Qu'est-ce qui est mal fait ?
                                    \item LE MAÎTRE Ce qu'on vous défend.
                                    \item L'ENFANT Quel mal y a-t-il à faire ce qu'on me défend.
                                    \item LE MAÎTRE On vous punit pour avoir désobéi.
                                    \item L'ENFANT Je ferai en sorte qu'on n'en sache rien.
                                    \item LE MAÎTRE On vous épiera.
                                    \item L'ENFANT Je me cacherai.
                                    \item LE MAÎTRE On vous questionnera.
                                    \item L'ENFANT Je mentirai.
                                    \item LE MAÎTRE Il ne faut pas mentir.
                                    \item L'ENFANT Pourquoi ne faut-il pas mentir ?
                                    \item LE MAÎTRE Parce que c'est mal fait, etc.
                                \end{itemize}
    \item Faites mieux : soyez raisonnable, et ne raisonnez point avec votre élève, surtout pour lui faire approuver ce qui lui déplait; car amener ainsi toujours la raison dans les choses désagréables, ce n'est que la lui rendre ennuyeuse, et la décréditer de bonne heure dans un esprit qui n'est pas encore en état de l'entendre. (Livre \rb{2})
    \item [EN PARLANT DES FABLES] prendront toujours le beau rôle; c'est le choix de l'amour-propre. (Livre \rb{2})
    \item [EN PARLANT DES FABLES] une horrible leçon pour l'enfance. (Livre \rb{2})
    \item l'appologue en les amusant, les abuse; séduits par le mensonge, ils laissent échapper la vérité [...] Les fables peuvent instruire les hommes; mais il faut dire la vérité nue aux enfants: sitôt qu'on la couvre d'un voile, ils ne se donnent plus la peine de le lever. (Livre \rb{2})
    \item Posons pour maxime incontestable que les premiers mouvements de la nature sont toujours droits: il n'y a point de perversité originelle dans le c\oe ur humain; il ne s'y trouve pas un seul vice dont on ne puisse dire comment et par où il y est entré. (Livre \rb{2})
    \item Ne donnez à votre élève aucune espèce de leçon verbale; il n'en doit recevoir que de l'expérience: ne lui infligez aucune espèce de châtiment, car il ne sait ce que c'est qu'être en faute: ne lui faites jamais demander pardon, car il ne saurait vous offenser. Dépourvu de toute moralité dans ses actions, il ne peut rien faire qui soit moralement mal, et qui mérite ni châtiment ni réprimande. (Livre \rb{2})
    \item Le chef-d'\oe vre d'une bonne éducation et de faire un homme raisonnable: et l'on prétend élever un enfant par la raison ! C'est commencer par la fin, c'est vouloir faire l'instrument de l'ouvrage. Si les enfants entendaient raison, ils n'auraient pas besoin d'être élevés; mais en leur parlant dès leur bas âge une langue qu'ils n'entendent point, on les accoutume à se payer de mots, à contrôler tout ce qu'on leur dit, à se croire aussi sages que leurs maîtres, à devenir disputeurs et mutins. (Livre \rb{2})
\end{enumerate}



\subsection{Soyinka}


\begin{enumerate}
    \item 
\end{enumerate}




\subsection{Andersen}


\begin{enumerate}
    \item 
\end{enumerate}



























\section{Valeurs (Respect / Vérité)}



\subsection{Rousseau}


\begin{enumerate}
    \item tous les engagements des enfants sont nuls par eux-mêmes, attendu que leur vue bornée ne pouvant s'étendre au-delà du présent, en s'engageant ils ne savent ce qu'ils font. (Libre \rb{2})
    \item Hors d'état de lire dans l'avenir, il ne peut prévoir les conséquences des choses; et quand il viole ses engagements, il ne fait rien contre la raison de son âge. (Livre \rb{2})
    \item La constance et la fermeté sont, ainsi que les autres vertus, des apprentissages de l'enfance; mais ce n'est pas en apprenant leurs noms aux enfants qu'on les leur enseigne, c'est en les leur faisant goûter, sans qu'il sachent ce que c'est. (Livre \rb{2})
    \item le mensonge de fait n'est pas naturel aux enfants; mais c'est la loi de l'obéissance qui produit la nécessité de mentir, parce que l'obéissance étant pénible, on s'en dispense en secret le plus qu'on peut, et que l'intérêt présent d'éviter le châtiment ou le reproche l'emporte sur l'intérêt éloigné d'exposer la vérité. (Livre \rb{2}) 
    \item L'aumône et une action d'homme qui connaît la valeur de ce qu'il donne, et le besoin que son semblable en a. L'enfant qui ne connaît rien de cela, ne peut avoir aucun mérite à donner. (Livre \rb{2})
\end{enumerate}



\subsection{Soyinka}


\begin{enumerate}
    \item 
\end{enumerate}




\subsection{Andersen}


\begin{enumerate}
    \item 
\end{enumerate}


















































\section{Education / Educateur / Nourrice / Jeux}



\subsection{Rousseau}


\begin{enumerate}
    \item L'éducation est un art, il est presque impossible qu'elle réussisse, puisque le concours nécessaire à son succès ne dépend de personne. (Livre \rb{1})
    \item Vivre, ce n'est pas respirer, c'est agir. (Livre \rb{1})
    \item Le devoir des femmes n'est pas douteux: mais on dispute si, dans le mépris qu'elles en font, il est égal pour les enfants d'être nourris de leur lait ou d'un autre. (Livre \rb{1})
    \item Ce que je crois voir d'avance est qu'un père qui sentirait tout le prix d'un bon gouverneur prendrait le parti de s'en passer. (Livre \rb{1})
    \item L'instruction des enfants est un métier où il faut savoir perdre du temps pour en gagner. (Livre \rb{2})
    \item Vous connaissez, dites-vous, le prix du temps et n'en voulez point perdre. Vous ne voyez pas que c'est bien plus perdre d'en mal user que de n'en rien faire, et qu'un enfant mal instruit est plus loin de la sagesse que celui qu'on n'a point instruit du tout. (Livre \rb{2})
    \item La constance et la fermeté sont, ainsi que les autres vertus, des apprentissages de l'enfance; mais ce n'est pas en apprenant leurs noms aux enfants qu'on les leur enseigne, c'est en les leur faisant goûter, sans qu'il sachent ce que c'est. (Livre \rb{2})
    \item il ne faut jamais infliger aux enfants le châtiment comme châtiment, mais qu'il doit toujours leur arriver comme une suite naturelle de leur mauvaise action. (Livre \rb{2})
    \item Je sais que toutes ces vertus par imitation sont des vertus de singe, et que nulle bonne action n'est moralement bonne que quand on la fait comme telle, et non parce que d'autres la font. Mais, dans un âge où le c\oe ur ne sent rien encore, il faut bien faire imiter aux enfants les actes dont on veut leur donner l'habitude, en attendant qu'ils les puissent faire par discernement et par amour du bien. (Livre \rb{2})
    \item Jeune instituteur, je vous prêche un art difficile, c'est de gouverner sans préceptes, et de tout faire en ne faisant rien. Cet art, j'en conviens, n'est pas de votre âge; il n'est pas propre à faire briller d'abord vos talent, ni à vous faire valoir auprès des pères: mais c'est le seul propre à réussir. (Livre \rb{2}) 
    \item Un père, quand il engendre et nourrit les enfants, ne fait de cela que le tiers de sa tâche. Il doit des hommes à son espèce, il doit à la société des hommes sociables; il doit des citoyens à l'Etat. Tout homme qui peut payer cette triple dette et ne le ffait pas est coupable. (Libre \rb{1})
\end{enumerate}



\subsection{Soyinka}


\begin{enumerate}
    \item 
\end{enumerate}




\subsection{Andersen}


\begin{enumerate}
    \item 
\end{enumerate}













































\section{Parole / Lecture / Inné / Acquis}



\subsection{Rousseau}


\begin{enumerate}
    \item Tout est bien sortant des mains de l'Auteur des choses, tout dégénère entre les mains de l'homme. (Livre \rb{1})
    \item L'inaptitude qu'on suppose aux enfants pour nos exercices est imaginaire, et que, si on ne les voit point réussir dans quelques-uns, c'est qu'on ne les y a jamais exercés. (Livre \rb{2})
    \item Apprenez-lui à parler uniment, clairement, à bien articuler, à prononcer exactement et sans affectation, à connaître et à suivre l'accent grammatical et la prosodie, à donner toujours assez de voix pour être entendu, mais à n'en donner jamais plus qu'il ne faut; défaut ordinaire aux enfants élevés dans les collèges: en toute chose rien de superflu. (Livre \rb{2})
    \item Parlez toujours correctement devant eux, faites qu'ils ne se plaisent avec personne autant qu'avec vous, et soyez sûrs qu'insensiblement leur langage s'épurera sur le vôtre sans que vous les ayez jamais repris. (Livre \rb{1})
    \item Je pose en fait qu'après deux ans de sphère et de cosmographie, il n'y a pas un seul enfant de dix ans qui, sur les règles qu'on lui a données, sût se conduire de Paris à Saint-Denis. Je pose en fait qu'il n'y en a pas un qui, sur un plan du jardin de son père, fût en état d'en suivre les détours sans s'égarer. Voilà ces docteurs qui savent à point où sont Pékin, Ispahan, le Mexique, et tous les pays de la terre. (Livre \rb{2})
\end{enumerate}



\subsection{Soyinka}


\begin{enumerate}
    \item 
\end{enumerate}




\subsection{Andersen}


\begin{enumerate}
    \item 
\end{enumerate}






































\section{Famille}



\subsection{Rousseau}


\begin{enumerate}
    \item Point de mère, point d'enfant. Entre eux les devoirs sont réciproques; et s'ils sont mal remplis d'un côté, ils seront négligés de l'autre. (Livre \rb{1})
    \item L'enfant doit aimer sa mère avant de savoir qu'il le doit. (Livre \rb{1})
    \item Il sera mieux élevé par un père judicieux et borné que par le plus habile maître du monde: car le zèle suppléera mieux au talent que le talent au zèle. (Livre \rb{1})
    \item Ce que je crois voir d'avance est qu'un père qui sentirait tout le prix d'un bon gouverneur prendrait le parti de s'en passer. (Livre \rb{1})
    \item Je voudrais même que l'élève et le gouverneur se regardassent tellement comme inséparable, que le sort de leurs jours fût toujours entre eux un objet commun. (Livre \rb{1})
\end{enumerate}



\subsection{Soyinka}


\begin{enumerate}
    \item 
\end{enumerate}




\subsection{Andersen}


\begin{enumerate}
    \item 
\end{enumerate}







\section{Autres citations}





\subsection{Rousseau}


\begin{enumerate}
    \item Tant qu'on reste dans le même état, on peut garder celles [les choses] qui résultent de l'habitude et qui nous sont le moins naturel. (Livre \rb{1})
\end{enumerate}



\subsection{Soyinka}


\begin{enumerate}
    \item 
\end{enumerate}




\subsection{Andersen}


\begin{enumerate}
    \item 
\end{enumerate}
















\end{document}