\documentclass[a4paper, 11pt, hidelinks]{article}
\usepackage{bookmark}
\usepackage[utf8]{inputenc} 
\usepackage[T1]{fontenc}
\usepackage{lmodern}
\usepackage{graphicx}
\usepackage[french]{babel}
\usepackage{geometry}
\usepackage{eucal}
\usepackage{caption}
\usepackage{float}
\usepackage{url}
\usepackage{amsmath}
\usepackage{amssymb}
\usepackage{color}
\usepackage{hyperref}
\usepackage{cancel}
\usepackage{tikz}
\usepackage{mathrsfs}  
\usepackage{esvect}
\usepackage[standard]{ntheorem}
\usepackage{romanbar}
\usepackage{titlesec}



\geometry{hmargin=2cm,vmargin=1.5cm}

\tikzset{
  treenode/.style = {shape=rectangle, rounded corners,
                     draw, align=center,
                     top color=white, bottom color=blue!5},
  root/.style     = {treenode, font=\Large, bottom color=red!10},
  env/.style      = {treenode, font=\ttfamily\normalsize},
  dummy/.style    = {circle,draw}
}

\newcommand{\prp}{\large \textbf{Proposition :} \large}

\newcommand{\tm}{\large \textbf{Théoreme :} \large}

\newcommand{\ex}{\textcolor{green}{Exemple :} }

\newcommand{\dm}{\textcolor{red}{\textbf{Démo :} } }

\newcommand{\de}{\large \textbf{Définition} \large }

\newcommand{\rmq}{\textbf{Remarque :} }

\newcommand{\bs}{\bigskip}

\newcommand{\voca}{\textcolor{blue}{\textbf{Vocabulaire} } }

\newcommand{\lem}{\textcolor{red}{\textbf{Lemme :} } }

\newcommand{\rb}[1]{\Romanbar{#1}}


\newcommand{\trinom}[3]{\begin{pmatrix}
    #1 \\
    #2 \\
    #3
\end{pmatrix}}

\newcommand{\quadrinom}[4]{\begin{pmatrix}
    #1 \\
    #2 \\
    #3 \\
    #4 \\
\end{pmatrix}}

\newcommand{\pentanom}[5]{\begin{pmatrix}
    #1 \\
    #2 \\
    #3 \\
    #4 \\
    #5
\end{pmatrix}}

\newcommand{\hexanom}[6]{\begin{pmatrix}
    #1 \\
    #2 \\
    #3 \\
    #4 \\
    #5 \\
    #6 
\end{pmatrix}}

\newcommand{\serie}[2]{\displaystyle\sum_{#1 =0}^{+\infty} #2_{#1} }

\newcommand{\tend}{\underset{n \to + \infty}{\longrightarrow} }

\newcommand{\Lra}{\Leftrightarrow}

\newcommand{\lra}{\leftrightarrow}

\newcommand{\Ra}{\Rightarrow}

\newcommand{\ra}{\rightarrow}

\newcommand{\la}{\leftarrow}

\newcommand{\La}{\Leftarrow}

\newcommand{\dsum}[2]{\displaystyle\sum_{#1}^{#2} }

\newcommand{\dint}[2]{\displaystyle\int_{#1}^{#2} }

\newcommand{\ntend}{\underset{n \to + \infty}{\not \longrightarrow} }

\newenvironment{lmatrix}{$ \left|\begin{array}{l} }{\end{array}\right.$}

\newcommand{\img}[4]{\begin{figure}[!ht]
    \centering
    \includegraphics[scale=#1 ]{#2}
    \caption{#3}
    \label{#4}
    \end{figure} }    
\begin{document}

\newcommand{\grad}[1]{\vv{grad}#1}


\title{Livret de citations}
\author{Schobert Néo}

\maketitle

\tableofcontents


\newpage














\section{Nature / Condition humaine}



\subsection{Rousseau}


\begin{enumerate}
    \item Tout est bien sortant des mains de l'Auteur des choses, tout dégénère entre les mains de l'homme. (Livre \rb{1})
    \item Nos têtes seraient mal de la façon de l'Auteur de notre être: il nous les faut façonner au-dehors par les sages-femmes et au-dedans par les philosophes. (Livre \rb{1})
\end{enumerate}



\subsection{Soyinka}


\begin{enumerate}
    \item 
\end{enumerate}




\subsection{Andersen}


\begin{enumerate}
    \item 
\end{enumerate}










\section{Etat d'enfance / Force / faiblesse}



\subsection{Rousseau}


\begin{enumerate}
    \item La race humaine eût périt si l'homme n'eût commencé par être enfant. (Livre \rb{1})
    \item Thétis, pour rendre son fils invulnérable, le plongea, dit la fable, dans l'eau du Styx. (Livre \rb{1})
    \item Un corps débile affaiblit l'âme. (Livre \rb{1})
\end{enumerate}



\subsection{Soyinka}


\begin{enumerate}
    \item 
\end{enumerate}




\subsection{Andersen}


\begin{enumerate}
    \item 
\end{enumerate}
























\section{Inné et acquis}



\subsection{Rousseau}


\begin{enumerate}
    \item Tout est bien sortant des mains de l'Auteur des choses, tout dégénère entre les mains de l'homme. (Livre \rb{1})

\end{enumerate}



\subsection{Soyinka}


\begin{enumerate}
    \item 
\end{enumerate}




\subsection{Andersen}


\begin{enumerate}
    \item 
\end{enumerate}


























\section{Clivage enfants / adultes}



\subsection{Rousseau}


\begin{enumerate}
    \item La race humaine eût périt si l'homme n'eût commencé par être enfant. (Livre \rb{1})

\end{enumerate}



\subsection{Soyinka}


\begin{enumerate}
    \item 
\end{enumerate}




\subsection{Andersen}


\begin{enumerate}
    \item 
\end{enumerate}
































\section{Société / Patrie}



\subsection{Rousseau}


\begin{enumerate}
    \item Tout patriote est dur aux étrangers; ils ne sont qu'hommes, il ne sont rien à ses yeux. (Livre \rb{1})
    \item Ces deux mots patrie et citoyens doivent être effacés des langues modernes. (Livre \rb{1})
\end{enumerate}



\subsection{Soyinka}


\begin{enumerate}
    \item 
\end{enumerate}




\subsection{Andersen}


\begin{enumerate}
    \item 
\end{enumerate}




























\section{Temps / Nostalgie}



\subsection{Rousseau}


\begin{enumerate}
    \item 
\end{enumerate}



\subsection{Soyinka}


\begin{enumerate}
    \item 
\end{enumerate}




\subsection{Andersen}


\begin{enumerate}
    \item 
\end{enumerate}




































\section{Imagination}



\subsection{Rousseau}


\begin{enumerate}
    \item 
\end{enumerate}



\subsection{Soyinka}


\begin{enumerate}
    \item 
\end{enumerate}




\subsection{Andersen}


\begin{enumerate}
    \item 
\end{enumerate}





























\section{Sens}



\subsection{Rousseau}


\begin{enumerate}
    \item 
\end{enumerate}



\subsection{Soyinka}


\begin{enumerate}
    \item 
\end{enumerate}




\subsection{Andersen}


\begin{enumerate}
    \item 
\end{enumerate}
































\section{Savoir et savoir-vivre}



\subsection{Rousseau}


\begin{enumerate}
    \item Faute de savoir se guérir, que l'enfant sache être malade: cet art supplée à l'autre, et souvent réussit beaucoup mieux; c'est l'art de la nature. \rb{1}
\end{enumerate}



\subsection{Soyinka}


\begin{enumerate}
    \item 
\end{enumerate}




\subsection{Andersen}


\begin{enumerate}
    \item 
\end{enumerate}














































\section{Tyrannie de l'enfance}



\subsection{Rousseau}


\begin{enumerate}
    \item Il faut qu'il donne des ordres ou qu'il en reçoive. Ainsi ses premières idées sont celles d'empire et de servitude. (Livre \rb{1})
\end{enumerate}



\subsection{Soyinka}


\begin{enumerate}
    \item 
\end{enumerate}




\subsection{Andersen}


\begin{enumerate}
    \item 
\end{enumerate}
















\section{Obéissance}



\subsection{Rousseau}


\begin{enumerate}
    \item Il faut qu'il donne des ordres ou qu'il en reçoive. Ainsi ses premières idées sont celles d'empire et de servitude. (Livre \rb{1})

\end{enumerate}



\subsection{Soyinka}


\begin{enumerate}
    \item 
\end{enumerate}




\subsection{Andersen}


\begin{enumerate}
    \item 
\end{enumerate}



























\section{Emancipation / Liberté}



\subsection{Rousseau}


\begin{enumerate}
    \item Vivre est le métier que je lui veux apprendre. En sortant de mes mains, il ne sera j'en conviens, ni magistrat, ni soldat, ni prêtre; il sera premièrement homme. (Livre \rb{1})
    \item Vivre, ce n'est pas respirer, c'est agir. (Livre \rb{1})
    \item Il n'y a qu'une seule chose à enseigner aux enfants: c'est celle des devoirs de l'homme. (Livre \rb{1})
\end{enumerate}



\subsection{Soyinka}


\begin{enumerate}
    \item 
\end{enumerate}




\subsection{Andersen}


\begin{enumerate}
    \item 
\end{enumerate}



































\section{Caprice}



\subsection{Rousseau}


\begin{enumerate}
    \item 
\end{enumerate}



\subsection{Soyinka}


\begin{enumerate}
    \item 
\end{enumerate}




\subsection{Andersen}


\begin{enumerate}
    \item 
\end{enumerate}

























\section{Vice et vertu / Paresse}



\subsection{Rousseau}


\begin{enumerate}
    \item 
\end{enumerate}



\subsection{Soyinka}


\begin{enumerate}
    \item 
\end{enumerate}




\subsection{Andersen}


\begin{enumerate}
    \item 
\end{enumerate}

























































\section{Vouloir et pouvoir}



\subsection{Rousseau}


\begin{enumerate}
    \item 
\end{enumerate}



\subsection{Soyinka}


\begin{enumerate}
    \item 
\end{enumerate}




\subsection{Andersen}


\begin{enumerate}
    \item 
\end{enumerate}


























































\section{Perversion}



\subsection{Rousseau}


\begin{enumerate}
    \item Tout est bien sortant des mains de l'Auteur des choses, tout dégénère entre les mains de l'homme. (Livre \rb{1})
\end{enumerate}



\subsection{Soyinka}


\begin{enumerate}
    \item 
\end{enumerate}




\subsection{Andersen}


\begin{enumerate}
    \item 
\end{enumerate}














































\section{Raisonnement / Logique}



\subsection{Rousseau}


\begin{enumerate}
    \item 
\end{enumerate}



\subsection{Soyinka}


\begin{enumerate}
    \item 
\end{enumerate}




\subsection{Andersen}


\begin{enumerate}
    \item 
\end{enumerate}



























\section{Valeurs (Respect / Vérité)}



\subsection{Rousseau}


\begin{enumerate}
    \item 
\end{enumerate}



\subsection{Soyinka}


\begin{enumerate}
    \item 
\end{enumerate}




\subsection{Andersen}


\begin{enumerate}
    \item 
\end{enumerate}


















































\section{Education / Educateur / Nourrice / Jeux}



\subsection{Rousseau}


\begin{enumerate}
    \item L'éducation est un art, il est presque impossible qu'elle réussisse, puisque le concours nécessaire à son succès ne dépend de personne. (Livre \rb{1})
    \item Vivre, ce n'est pas respirer, c'est agir. (Livre \rb{1})
    \item Le devoir des femmes n'est pas douteux: mais on dispute si, dans le mépris qu'elles en font, il est égal pour les enfants d'être nourris de leur lait ou d'un autre. (Livre \rb{1})
    \item Ce que je crois voir d'avance est qu'un père qui sentirait tout le prix d'un bon gouverneur prendrait le parti de s'en passer. (Livre \rb{1})
\end{enumerate}



\subsection{Soyinka}


\begin{enumerate}
    \item 
\end{enumerate}




\subsection{Andersen}


\begin{enumerate}
    \item 
\end{enumerate}













































\section{Parole / Lecture}



\subsection{Rousseau}


\begin{enumerate}
    \item 
\end{enumerate}



\subsection{Soyinka}


\begin{enumerate}
    \item 
\end{enumerate}




\subsection{Andersen}


\begin{enumerate}
    \item 
\end{enumerate}






































\section{Famille}



\subsection{Rousseau}


\begin{enumerate}
    \item Point de mère, point d'enfant. Entre eux les devoirs sont réciproques; et s'ils sont mal remplis d'un côté, ils seront négligés de l'autre. (Livre \rb{1})
    \item L'enfant doit aimer sa mère avent de savoir qu'il le doit. (Livre \rb{1})
    \item Il sera mieux élevé par un père judicieux et borné que par le plus habile maître du monde: car le zèle suppléera mieux au talent que le talent au zèle. (Livre \rb{1})
    \item Ce que je crois voir d'avance est qu'un père qui sentirait tout le prix d'un bon gouverneur prendrait le parti de s'en passer. (Livre \rb{1})
    \item Je voudrais même que l'élève et le gouverneur se regardassent tellement comme inséparable, que le sort de leurs jours fût toujours entre eux un objet commun. (Livre \rb{1})
\end{enumerate}



\subsection{Soyinka}


\begin{enumerate}
    \item 
\end{enumerate}




\subsection{Andersen}


\begin{enumerate}
    \item 
\end{enumerate}







\section{Autres citations}





\subsection{Rousseau}


\begin{enumerate}
    \item Tant qu'on reste dans le même état, on peut garder celles [les choses] qui résultent de l'habitude et qui nous sont le moins naturel. (Livre \rb{1})
\end{enumerate}



\subsection{Soyinka}


\begin{enumerate}
    \item 
\end{enumerate}




\subsection{Andersen}


\begin{enumerate}
    \item 
\end{enumerate}
















\end{document}