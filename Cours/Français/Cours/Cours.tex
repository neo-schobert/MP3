\documentclass[a4paper, 11pt, hidelinks]{article}
\usepackage{bookmark}
\usepackage[utf8]{inputenc} 
\usepackage[T1]{fontenc}
\usepackage{lmodern}
\usepackage{graphicx}
\usepackage[french]{babel}
\usepackage{geometry}
\usepackage{eucal}
\usepackage{caption}
\usepackage{float}
\usepackage{url}
\usepackage{amsmath}
\usepackage{amssymb}
\usepackage{color}
\usepackage{hyperref}
\usepackage{cancel}
\usepackage{romanbar}
\usepackage{titlesec}

\geometry{hmargin=2cm,vmargin=1.5cm}

\newcommand{\dstylesum}{\displaystyle\sum}

\newcommand{\dstyleprod}{\displaystyle\prod}

\newcommand{\prp}{\large \textbf{Proposition :} \large}

\newcommand{\tm}{\large \textbf{Théoreme :} \large}

\newcommand{\ex}{\textcolor{green}{Exemple :} }

\newcommand{\dm}{\textcolor{red}{\textbf{Démo :} } }

\newcommand{\de}{\large \textbf{Définition} \large }

\newcommand{\rmq}{\textbf{Remarque :} }

\newcommand{\bs}{\bigskip}

\newcommand{\voca}{\textcolor{blue}{\textbf{Vocabulaire} } }

\newcommand{\cit}{\large \textcolor{blue}{\textbf{Citation :}} \large }

\newcommand{\rb}[1]{\Romanbar{#1}}
\newcommand{\trinom}[3]{\begin{pmatrix}
    #1 \\
    #2 \\
    #3
\end{pmatrix}}

\newcommand{\quadrinom}[4]{\begin{pmatrix}
    #1 \\
    #2 \\
    #3 \\
    #4 \\
\end{pmatrix}}

\newcommand{\pentanom}[5]{\begin{pmatrix}
    #1 \\
    #2 \\
    #3 \\
    #4 \\
    #5
\end{pmatrix}}

\newcommand{\hexanom}[6]{\begin{pmatrix}
    #1 \\
    #2 \\
    #3 \\
    #4 \\
    #5 \\
    #6 
\end{pmatrix}}

\newcommand{\serie}[2]{\displaystyle\sum_{#1 =0}^{+\infty} #2_{#1} }

\newcommand{\tend}{\underset{n \to + \infty}{\longrightarrow} }

\newcommand{\Lra}{\Leftrightarrow}

\newcommand{\lra}{\leftrightarrow}

\newcommand{\Ra}{\Rightarrow}

\newcommand{\ra}{\rightarrow}

\newcommand{\la}{\leftarrow}

\newcommand{\La}{\Leftarrow}

\newcommand{\dsum}[2]{\displaystyle\sum_{#1}^{#2} }

\newcommand{\dint}[2]{\displaystyle\int_{#1}^{#2} }

\newcommand{\ntend}{\underset{n \to + \infty}{\not \longrightarrow} }

\newenvironment{lmatrix}{$ \left|\begin{array}{l} }{\end{array}\right.$}


\setcounter{secnumdepth}{4}

\titleformat{\paragraph}
{\normalfont\normalsize\bfseries}{\theparagraph}{1em}{}
\titlespacing*{\paragraph}
{0pt}{3.25ex plus 1ex minus .2ex}{1.5ex plus .2ex}



\begin{document}




\title{L'enfance dans Ake, les années d'enfance}
\author{Schobert Néo}

\maketitle

\tableofcontents


\newpage




\section{Un récit autobiographique centré autour du personnage de Wole}


\subsection{Un récit autobiographique}


\de Récit \textbf{rétrospectif} en prose qu'une personne réelle fait de \textbf{sa propre existence}, lorsqu'elle met l'accent sur sa vie individuelle,
en particulier sur l'histoire de sa personnalité.
L'auteur, le narrateur et le personnage sont \textbf{confondus} dans cette première personne qui est le \textbf{"je"}.

\bigskip
Le roman d'apprentissage est un \textbf{genre littéraire} défini comme l'histoire d'un individu qui va dans le monde pour apprendre à se connaitre
et qui cherche des aventures pour se découvrir et pour donner la pleine mesure de sa personnalité.

\smallskip
Wole rapporte \textbf{ses} souvenirs; mais aussi ceux des \textbf{autres} (en particulier de sa mère).







\subsection{La mémoire}

\subsubsection{Une mémoire sensorielle}



Ce que l'auteur cherche à faire ressentir, c'est "l'invisible réseau de la personnalité d'Aké". (p285)

\bs

\begin{itemize}
    \item On nous parle du "spectacle des odeurs". (p290) (y compris des odeurs désagréables)

          Toutes les \textbf{odeurs} se mêlent. (les odeurs de nourriture, d'urine, des femmes)


    \item Au-delà des odeurs, il y aussi les \textbf{sons} qui importent. (les vois)

          \cit "Des voix intimes, discrètes, des êtres et des choses qui remplissaient ensemble Aké de l'aube au crépuscule." (p284)


    \item Il y a aussi les \textbf{couleurs} à compter parmi les éléments sensorielle.

          \cit "La poussière la plus rouge qu'il soit possible de trouver sur toute la surface de la Terre." (p250)


    \item Et il y a aussi la \textbf{nourriture}.

          \cit "La nourriture du marché." (p294)
\end{itemize}



\subsubsection{Le constat douloureux de l'écoulement du temps et des changements qu'il occasionne}



\cit "Le mystère a été chassé" (p16) citation qui évoque le désenchantement lié à l'écoulement du temps.

Les choses changent certes mais souvent ce changement est vécu comme un \textbf{désenchantement}. Ici, ce qui disparait avec le temps,
c'est la \textbf{magie} (le mystère). C'est souvent très nette avec l'enfance.
\bs

\cit "A l'époque dont je parle" (p16). Il utilise souvent le \textbf{"ne que"} qui dévoile le vision \textbf{pessimiste} du changement. ("Ce n'est plus qu'une épave"...)

\bs
\cit "Il est arrivé malheur à la mission d'Aké." (p16) Ce désenchantement est même explicite ici.

La voiture a toujours été une épave mais il ne s'en appercoit qu'en étant adulte. Pour les enfants, cette voiture
était un "dragon".


\cit "Ses yeux se sont changés en orbite rouillé, son visage de dragon s'est effondré, ses dents sont peu à peu tombées." (p16)


\bs

Ce désenchantement est \textbf{subjectif} mais aussi \textbf{objectif}. Par exemple, le verger à disparu. Certaines choses deviennent terne car elles sont
moins bien entretenues. \textbf{On magnifie le passé} en général: c'est la \textbf{nostalgie} de l'enfance. Cela est le propre du passage
à l'âge adulte. Vieillir c'est adapter une nouvelle perspective de réalité. Cette nouvelle perspective est marquée par \textbf{la perte de l'extraordinaire}.

\bs
\cit "Les perspectives élargies d'une enfance disparu."


\cit "Même la distance qui sépare le campanile du clôcher a raccourci."

\bs
Grossomodo, rien n'a changé; ce sont les \textbf{éléments intimes} qui changent:


\cit "Rien de tout cela n'a changé" En disant ça, il ne \textbf{se contredit pas}, il parle globalement


\cit "Mais ce n'est pas le cas des choses plus intimes"







\subsection{Wole: Ecrivain, narrateur et personnage}
\subsubsection{Description de Wole}


\paragraph{a) Un enfant curieux et "raisonneur" (\rb{13}, p354)}


\begin{itemize}
    \item \textbf{La curiosité}

          Il est curieux de ce qui l'entoure aussi pour mieux le maîtriser.

          On retrouve aussi cela (p314) quand au lycée d'Abeokuta il tente d'éviter le bizutage.

          \cit "Je me mis instinctivement à étudier de très près mes nouveaux compagnons et à rechercher les voies et moyens de survivre au milieu d'eux" (\rb{12}, p314)
    \item \textbf{Les questions}

          Cela amène Wole à poser beaucoup de questions: et cela agace les adultes.

          \cit "On dit que [...] c'est une autre question" (p263 et 276)

          Ce sens du débat on dit qu'il le tient de sons père ce qui énèrve sa mère qui le trouve "trop raisonneur" (p110 et 113). On l'appelle "Mr l'avocat".

    \item \textbf{Les livres}

          Wole va puiser sa curisoté dès qu'il peut dans les livres "Les livres sont un trésor." (p158)
\end{itemize}

\paragraph{b) Un enfant rêveur: "il a toujours été porté à rêvasser" (\rb{7}, p197)}


Pour chrétienne sauvage, la rêverie est une maladie. Elle est inquiète quant à la capacité de l'enfant à s'extraire de ses rêves.
Elle appelle ça la \textbf{rumination}.

Quand Wole déconnecte de la réalité, ça se traduit par des accès de violences. (Voir le massacre des roses ou du petit frère.)
A se moment d'ailleurs, Chrétienne sauvage le défend en disant que Wole était ailleurs: il rêvait.


Chrétienne sauvage met en garde contre les dangers de la rêverie:

\cit "Il faut absolument faire quelque chose avant que tu ne te tues ou que tu mettes feu à la maison" (\rb{5}, p 152)


\cit "Ce trou où je ne me souvenait pratiquement plus de rien"

\paragraph{c) Un enfant turbulent}



Il fait preuve de mauvaise volonté \textbf{(d'èmi èsù)}. Au yeux de la mère, c'est un crime impardonnable.


\subsubsection{La famille de Wole}


\cit "Un besoin d'être avec la famille, de partager l'intimité tranquille du toucher, des regards, dans un rapprochement palpable en chacun de nos actes." (\rb{11}, p308)


\paragraph{a) Le père}


La maison de ses parents est un \textbf{endroit protecteur} pour Wole.

\cit "Pleine de gâteries imprévues" (\rb{5}, p135)

\bs
Ces lieux sont des lieux très protecteurs.

\cit Il y là bas, "un amour, une protection [...] plus solide, plus vieille que la terre." (\rb{5}, p135)

\cit"A Isara c'était un sentiment constant, indiscutable, et que rien ne pouvait menacer d'ébranler." (C'est ici un sentiment de protection) (\rb{9}, p267)
\bs

Son père est raffiné; c'est un intelectuel qui aime débattre. Cet aura d'intelectuel lui est donné par son poste de directeur de l'école. (HM: Head master)

\cit "Quant à mon père, évidemment, je le tenais pratiquement comme invulnérable" (\rb{1}, p35)

\smallskip
Dans ses joutes orales, il va même jusqu'à débattre autour de la \textbf{religion} :

\cit "Mon père avait l'habitude de parler comme s'il était à tu et à toi avec Dieu." (\rb{1}, p48)

\smallskip
Cela le rend \textbf{intransigeant} et \textbf{redoutable} "personne n'échappait à Essay." (p147)

A force d'opiniâtreté et d'intransigeance il devient "un personnage étrange qui poursuivait des exigences impossibles avec une telle opiniâtreté." (p142).





\paragraph{b) La mère}



Les enfants ont de nombreuses mères. Parmi toutes ces mères, une sort du lot et semble aimante, il s'agit de \textbf{la femme du Libraire}.
Elle s'oppose généralement à ce que subissent les enfants.

\bs

Les femmes récupèrent les enfants comme s'ils étaient leurs.
Chrétienne sauvage aussi receuille un certain nombre d'enfants.

\cit "C'était elle et non les parents, qui semblait avoir donné naissance à cette famille, l'avoir vomie puis résorbée pour en faire des enfants de la maison" \rb{6} p157
\bs

Même les serviteurs de la maison l'appelle maman. C'est \textbf{la mère de la maison}; pas seulement des enfants.
De même, les mots pères et mères peuvent signifier grands-parents.

\bs
Logiquement, si des mères récuperent des enfants, d'autres démissionnent.

\bs
Chrétienne sauvage est un \textbf{oximore} pour plusieurs raisons:
\begin{itemize}
    \item Société / sauvage
    \item Amour chrétien / sauvage
\end{itemize}
\bs

Wole découvre à la fin que Chrétienne sauvage est opposée à la violence.

\cit "Une incroyable découverte: chrétienne sauvage abhorrait la violence! c'était une révélation stupéfiante." (\rb{14}, p403)




\paragraph{c) La fratrie: "le monde"}


Wole a:

\begin{itemize}
    \item Une s\oe ur qui s'appelle \textbf{Tinu}.
          \bs

          \cit "Tinu était ma plus proche camarade de jeu, et un lien de protection mutuel s'était établi entre nous, lien qui ne se manifestait jamais que
          lorsqu'elle était blessée ou menacée" (\rb{6}, p159)
          \bs

    \item Il a un frère cadet qui s'appelle \textbf{Ladipo} qui va changer de nom car il est trop casse-cou (Femi).
    \item Sa s\oe ur \textbf{Folasade}.
\end{itemize}


\bs
A cette fratrie s'ajoute tous les petits abandonnés qui rejoignent la maison.






\section{Le temps de l'enfance}


\subsection{Le récit évoque des évènements situés dans les années 40}
Ce temps de l'enfance est très identifié, et est mis en relief à travers les yeux d'enfant.



\subsubsection{Des évènements en lien avec l'histoire mondiale}



\paragraph{La référence à la Deuxième Guerre Mondiale}


On a une référence presque fantasmagorique à Hitler, qui envahit l'imaginaire des enfants, et devient
un personnage fantastique (échappe au réel) et menaçant.
Par exemple, dans les joutes verbales avec le père et le libraire, Hitler est présenté comme une menace
comparable au diable, brandit comme frayeur absolue.
\smallskip

Il rend compte de cette globalisation avant l'heure :

\cit "La nouvelle que Hitler se lançait à la conquête du monde commençait à nous parvenir". (\rb{7}, p199)

\bs

Cette menace globale qui s'étend imprègne le vécu individuel et même la famille, avec par exemple le
surnom qu'on donne à Dipo, surnommé par la femme du parrain de Wole : "allmani", allemand.


\cit "la race allemande avait acquis une réputation guerrière et redoutable". (\rb{7}, p199)


\cit "si lointaine qu'était cette guerre … Hitler se rapprochait de chez nous de jour en jour". (\rb{8}, p211)

\bs
On commence à en entendre parler, il se rapproche :

\cit "vraiment Hitler se rapprochait et personne ne savait ce qu'il ferait s'il apparaissait" (\rb{7}, p 212)

\bs
Il y a une construction d'une menace incarnée dans une figure. Paa adatan (un fou) a une réponse
(et manie une langue étrange pour l'enfant), il est prêt à le combattre, mais on se demande souvent
s'il arrive, comme quand l'oncle de Dipo arrive et que les gens le prennent pour Hitler. (p 163)
(date : 1938).

\bs

Les parents sont aussi très inquiets et se préparent à la possibilité de la guerre sur leur territoire,
donc on passe les vitres noires, on allume pas la nuit pour pas être visible, et (211)
le mot d'ordre gagner la guerre investit tout, il y a des coiffures gagner la
guerre, on se dit gagner la guerre dans le quotidien, ça rentre dans le quotidien.
\bs

Autre référence à Hiroshima et Nagasaki (419), Mussolini (427), (12.329) : camps de concentration (donc
après qu'on connaisse leur existence).


\paragraph{Repères culturels}

Il y a une référence à des chanteurs africains de l'époque (p 110) et une référence à la radio anglaise (p 110),
donc on s'ancre dans une réalité culturelle en plus d'historique.


\subsubsection{Des repères autobiographiques}


\begin{itemize}
    \item p56 : Wole veut commencer l'école et il a moins de 3 ans.
    \item p99 : il a 4 ans et demi,
    \item p272: incisions rituelles faites au poignet et aux chevilles : il a 8 ans et demi,
    \item p354 : premier entretien pour le lycée national, il a 10 ans, y rentre à 11 ans, et la dernière phrase est la fin de l'enfance :


          \cit "le moment était venu d'entreprendre les mutations mentales nécessaires pour accéder à un nouvel univers
          d'adulte irrationnel et à leur discipline". (\rb{15}, p432)


          Il atteint alors un univers nouveau, nécessité voire contrainte d'adaptation, et ça n'a pas l'air ouf :
          irrationnel et discipline, semble plutôt marqué par un désenchantement (regard de l'homme adulte qui
          n'est plus enfant).
\end{itemize}

\bs
On pourrait penser qu'irrationnel et discipline sont opposés, cela voudrait dire que le monde de
l'enfance est celui de la liberté (ok) et rationalité (bof) parce que pour Wole les adultes font
un peu n'importe quoi, il les trouve toujours incohérents, rentrer dans ce monde contraste alors
avec l'enfant qui appréhende le monde de manière moins incohérente et moins irrationnel.

\subsection{Le contexte colonial}


\subsubsection{La colonisation}

La colonisation semble pacifique pendant tout le quotidien, le lieu n'était pas un lieu de
conflictualité, et dans la mémoire de l'enfant ce n'est pas un univers de violence, la violence
est plus de la mère que du colon. Le Nigéria a été colonisé tôt, ce qui superpose les sources de
pouvoir, d'autorité.

\bs

Il y a d'abord le pouvoir colonial dans la capitale, et localement le district
officer, et les pouvoirs locaux : l'alake abe okouta (le roi), entouré de son conseil de chefs, les
ogboni, une institution entre le sage, la mémoire de la communauté, le politique, entre un héritage
traditionnel et ont une autorité politique, sont entre le politique et le spirituel, décrits entre
(p 382-383), groupe un peu mystérieux qui a plus de pouvoirs que le roi.



\paragraph{La propagande coloniale}


On est dans un univers qui n'est pas violent mais qui est de domination, et qui dit domination dit
propagande, qu'on remarque quand béré couti va en grande Bretagne et dit que les femmes dans les colonies
vont pas bien, et il y a toute une propagande qui vise à donner une bonne image des femmes dans les
colonies, fait témoigner des femmes sous la pression pour leur faire dire que tout va bien, mais elles
vont imprimer des articles qui montrent l'inverse.



\paragraph{L'opposition au colonisateur}

Dans une chanson populaire, on dit que la radio est à la solde des blancs: la radio est au service
des "mensonges de l'homme blanc". (\rb{7}, p209)


Dans le discours un peu fou de paa datan, il dit que les blancs ne veulent pas l'enrôler pour battre
Hitler parce qu'il dit que les blancs ne veulent pas qu'un noir gagne la guerre.

\bs

Au départ, combat pour des problèmes très locaux et personnels, se mobilisent sur des questions comme
la mortalité infantile et l'analphabétisme des femmes qui ne leur donne pas le bagage pour aider dans
l'éducation.
\bs

Au début, très local mais donne un grand soulèvement, qui, à partir des ces problèmes
quotidiens, disent qu'elles sont dans la misère, et à la base de cette misère, il y a les taxes et le
cout de la vie.


Font une pétition pour l'abolition d'un certain nombre de taxes, et leur mouvement
devient un syndicat : le syndicat des femmes ekba, avec une terminologie marxiste, lutte pour les
femmes exploitées (p 352), et chrétienne sauvage est très engagée dans ce mouvement.


Ce mouvement devient national : syndicat des femmes nigériennes, qui devient un mouvement d'opposition à
la colonisation (p 377).
\bs


Il a des répercussions locales : les femmes vont marcher sur le palais et l'alake, et s'opposent aux
ogboni, réclament à l'alake qu'il prenne en compte leurs problèmes, et devient un mouvement anticolonial
parce que refusent de parler avec le district officer: "ce blanc insolent" qui les insulte, et on passe
d'un discours de la misère à un discours anticolonial, réclamant que les colons partent du palais (p 395).


Les ogboni essaient de résister, quand les hommes résistent ils renvoient les femmes aux clichés sur le
femmes.
\bs


Le retournement vient d'un pouvoir surnaturel, le gars qui dit ça s'effondre, les femmes
s'attaquent aux ogboni, l'alake tombe, et il y a un affront avec le district officer, beere dit :


\cit "je connais votre mentalité à vous les blancs,…", (p420)


\cit "les hommes blancs sont racistes, tu connais leur histoire…". (p425)
\bs

Tout ce mouvement très important dans la décolonisation, agrégé par le mouvement des
femmes, et rencontre des figures historiques de cette décolonisation comme le premier président du
Nigéria indépendant Zik, avec un discours et une rhétorique de décolonisation "on exigeait des blancs
qu'il nous laissent nous diriger nous-mêmes", et on aboutit à l'indépendance du Nigéria et du Cameroun.


\subsubsection{Les rapports avec les Blancs}

Il s'agit ici d'un rapport ambivalent.

\paragraph{Pour Wole, les Blancs sont une source d'étonnement}


Wole croit que les blancs sont des noirs dépigmentés. (albinos)
Wole trouve le blanc bizarre; il a un accent "invraissemblable" (p93)
Les blancs ont une particularité très amusante pour Wole: ils portent des chaussures.

\bs

Il y a une reconnaissance entre les noirs et le blancs; d'après le grand-père de Wole:


\cit "Les gens ont eu les yeux ouverts par les blancs" (\rb{9}, p273)

Les blancs ont été "formés à l'ombre".

Cette reconnaissance est aussi marquée par le père de Wole écoute la radio anglaise et il
veut confier l'éducation de Wole au lycée national tenu par des blancs.
\bs


A l'inverse, il existe aussi de mépris entre les blancs et noirs.


En effet, Beere Kuti décrète que celui qui croyait que les bons morceaux devaient être laissés aux enfants
était un imbécile ou un anglais. (\rb{10}, p298)

\bs

En dehors de cela, les nigérians s'approprient des pratiques importées par les blancs.


Scène du mariage dont le sérémonial n'est pas nigérian mais occidental. Cette scène de mariage fait éprouver
à l'enfant une sorte de malaise; quelque chose de triste. A cause d'un effet d'empreint artificiel "d'une apparence
etrangère, étrangère par dessus le marché" (p338)



C'est tout le raisonnement de  Ransom Kuti qui va en fait critiquer l'éducation des blancs.
\cit "il doutait de la capacité des professeurs blancs de donner une éducation valable aux africains"
(\rb{73}, p362)
Pour Ransome Kuti, les blancs ne savent pas former le caractère; ils affaiblissent le caractère des enfants.
Ils n'utilisent pas assez la baguette.

\cit "le blanc est une créature étrange" (p 426)

\bs


Le regards des noirs témoigne d'un certain mépris, mais aussi d'une certaine méfiance:
Les blancs ne veulent pas former les noirs car ils n'en ont aucun intérêt: un noir leur couperet la tête
une fois formé.(p365)





\paragraph{La reconnaissance des apports de Blancs}








\paragraph{Mais l'influence des Blancs est aussi rejetée}





\subsubsection{Un syncrétisme culturel}



Se côtoient la foi chrétienne et les croyances traditionnelles.

\bs

Chrétienne sauvage est chrétienne et semble même avoir une foi aveugle (croit aux miracles)

Elle est attachée aux croyances traditionnelles aussi (mais ne croit pas à n'importe quoi non plus).

Elle ne croit pas par exemple au fait que manger un lit de mantee religieuse guérit des fuites nocturnes. (\textbf{énurésie})
\bs


Pour soigner aussi, on utilise autant les remèdes traditionnels que les remèdes médicaux.
On mélange parfois même les deux.



\subsection{Tradition et modernité}


\subsubsection{La tradition}


\paragraph{Le village paternel d'Isara}

Ce village est ce qu'on peut appeler "le monde d'avant".

\bs

Il s'agit du village où se trouve la maison de Wole:


Elle se situe "à plusieurs pas en arrière dans le passée" (p 134)


\cit "Le temps baignait tous les recoins, la patine des ancrêtres brillait
sur tous les objets, sur touts les visages." (p 134)
\bs


Le grand-père appartient au monde traditionnel opposé au monde chrétien; il incarne la sagesse.

Ce grand-père est attentif et met Wole en garde concernant le lycée. Wole étant précosse est plus jeune que les autres.
Ce lycée sera un "champ de bataille" (p274) selon le grand-père.
Il dit :

\cit "Les humains sont ceux qu'ils sont. Certains sont bons, d'autres sont méchants. Et il y en a qui deviennent méchants
simplement parce qu'ils sont poussés à bout. L'envie. Hmmm, il ne faut pas que tu commettes l'erreur de
croire que l'envie n'est pas un mobile puissant chez beaucoup. C'est une maladie que tu trouveras partout,
oui, partout." (il s'agit d'une sorte de leçon de vie) (\rb{9}, p275)









\paragraph{Les superstitions}

Isara, c'est aussi le lieu des scarifications rituelles. Il ratache l'enfant à une tradition mémoriale consistant
à marquer le corps afin qu'il appartiennent à un tout.

\bs

Quand un nouveau roi accède à la royauté, il doit manger le coeur et le foi du défunt roi. (p288)
concernant la $\underbrace{magie \;noire}_{faite\; pour\; etre\; nuisible\; aux\; autres}$ aussi : par exemple, les gens d'Ibeju sont considérés comme des empoisonneurs.
    



\paragraph{La magie noire}


Beaucoups de traditions ici dont :


Il sont connus aussi pour leur maîtrise des procédés magiques. On parle aussi de charme magique de certains pour réussir
leurs examens. Beaucoup d'exemples d'utilisation de magie (flemme de marquer) $\sim$ p250/280



\subsubsection{La modernité}



\paragraph{La modernité contemporaine des années 1940}

Les informations témoignent de cette modernité:

"La radio c'est l'Oracle ou la Voix". Le salon devient un sanctuaire. Les visages qui écoutent sont en extase;
le père devient un prêtre.
\bs



Les avions aussi témoignent de cette modernité:

Il sont tantot perçu comme objets de la fin du monde et tantot comme miracles volants.


Le passage aux années 1980 est perçu comme un désenchantement chez Wole.




\paragraph{La modernité contemporaine des années 1980}


Il y a à ce moment une grande critique envers les objets du monde occidental et envers la société de consommation:

\cit "Les produits d'une industrie mondiale de gâchie." (p287)

\bs

Comme beaucoup, Wole est partisan du c'était mieux avant. C'est dû à la nostalgie de l'enfance et au fait que 
lorsque l'on est jeune, on vit les choses pour la première fois $\to$ plus exitant donc.
\bs


Aussi, Wole critique la nourriture occidentale faite de toute sorte de matière caoutchouteuse.
\bs

La culture traditionnelle a laissé place à la culture automatique.


\subsection{L'appréhension du temps par Wolé}


\subsubsection{Le temps ne s'écoule pas de la même façon pour les enfants et pour les adultes}


L'adulte voit le temps long. Chez l'enfant, tout semble se passer vite.


\cit "Le goyavier possédait cette assurance indéfinissable d'avaler le temps, d'en supprimer l'existence." (p130)


\subsubsection{Un temps duel : entre rituel et changements}


Le temps est marqué par une routine; la répétition d'un rituel quotidien (p51)

\bs

De temps à autres, cette routine est stoppée par des changements soudains. (p183)


Les choses changent car le temps passe. Les gens aussi car ils vieillissent.

La question qui émèrge est la suivante:

Sommes-nous les mêmes lorsque l'on vieillit ?
\bs


Il y a de gros changements définifs tout de même, qui bouleversent les habitudes 


Par exemple, quand Dipo va naitre, la mère se met à grossir. 


C'est aussi la perte (ou la mort) qui vient perturber la routine.

Comme la perte de Folassade; la soeur. Le plus éffrayant c'est qu'alors, rien ne semble vraiment changer. Le monde
ne change pas malgrès le cataclysme. $\to$ C'est comme si ce n'était pas important qu'elle soit morte.


Certains aussi disparaissent sans qu'on s'en apperçoivent. (comme Adan, Toro anké ...)



\section{Les lieux de l'enfance}

\subsection{Aké}


\subsubsection{La mission et la maison}


\paragraph{Un espace protégé}


Il s'agit d'un espace entouré de murs qui délimité le terrain des humains de celui des esprits.


Les murs de la mission "donnaient l'impression d'un forteresse" (p 13).
Dans cette forteresse, les enfants se sentent "en sécurité" (p13)



\paragraph{Un espace édénique}


Il s'agit d'un espace édénique avec une référence au Vergers.(p15) Le grenadier est le vrai fruit de la Bible (p15).

On a donc de vraies références à l'eden.

Au contact de cette nature généreuse, l'enfant renoue avec "le monde illustré des Belles Histoires de la Bible" (p15)

\bs

Enfin, c'est un espace sacré:

Les arbres poussent "miraculeusement". Qui dit sacré dit aussi secret:

C'est un espace où l'enfant peut se cacher.



\paragraph{Mais: même dans cet espace renferme une part de menace}

Etant enfant, certaines choses peuvent prendre des caractéristiques importantes.

Ce ne sont pas de vraies menances en soie; mais des dangers pour l'enfant. (comme la falaise abrupt ou 
le fait que certains bosquet abritent des serpents monstruits (p13)).




\paragraph{Un espace pétri d'imaginaire}

On parle notamment du rocher. (p129) 

Grâce à l'imagination, le rocher devient une réalité biblique investit de mystère.


Sur ce rocher va se gréffer aussi "un monde de fables, de fantaisies, celui de la lampe d'Aladin et du Césame ouvre-toi"


Il renferme : "un monde intérieur d'esprit bienveillant" (p130).


\subsubsection{Le reste du monde}


On s'atarde peu du reste du monde; il est peu investit dans ce tome autobiographique.

L'enfant découvre le "reste" en grandissant.
\bs


Le reste du monde est plutôt profane. Les colines autour du villages sont qualifiées de "hauteurs impies".

Ce qui est important c'est la manière dont l'enfant va être amené à réviser ses connaissances 
de l'espace. Tout est plus grand dans le souvenir que dans la réalité.


Grandir, finalement, c'est redonner aux choses "leurs formes et leurs dimensions véritables" (p82) (pas très enchantant donc)


\section{Enfants et adultes}

\subsection{Le regard de l'enfant sur le monde}


L'enfant va émettre des idées; des vérités sur les choses et sur les adultes qui sont fausses.

Par exemple, "le sucre, c'est le test infaillible du fruit véritable." L'enfant admet comme vérité que $\forall fruit, fruit \in P(sucr\acute{e})$.

\bs

L'enfant pose des vérités mais nous lecteurs ou adultes les savons fausses.
\bs







\subsection{Les adultes vus à travers les yeux de l'enfant}


\subsubsection{Les êtres étranges}

\paragraph{L'assimilation des adultes à des animaux ou à des objets témoigne de leur étrangeté aux yeux de l'enfant}


De même, l'enfant pose des vérités sur l'adulte mais qui étonnement, se révèlent souvent être vraies.


L'enfant assimile l'adulte à un être mécanique. Les adultes semble êtres des animaux.


Les fidèles sont des souris d'Eglise.(p34) Le libraire a des gestes d'oiseaux (d'un rapace inquiétant) (p37)


\paragraph{Des comportements étranges}


Les adultes ont des comportements bizarres en particulier quand ils manifestent des sentiments excessifs.

Exemple: retour de la femme du mec. (p66)

\bs

Les adultes ont des mots étranges aussi.

\paragraph{Des adultes "magiques"}


La dextérité des adultes semble être magique (le couteau du boucher ou le tambour)


Chrétienne Sauvage aussi est magique:

\begin{itemize}
    \item Elle surgit en permanence comme par enchantement.
    \item Elle sait directement à qui appartient le pipi la nuit.(p66)
    \item Elle a de mysterieux présentiments inspirés. (p78)
\end{itemize}



\paragraph{Les secrets des adultes}

Comme l'enfant manque d'un savoir, il l'interprète comme quelque chose de caché (comme un secret).


C'est le cas de la maladie de son père qui semble être grave. Et puis cette menace disparait. 

\bs

La sexualité aussi est quelque chose de secret. (p376)

Les rites secrets des relations entre mari et femme.


\subsubsection{Les failles des adultes}

\paragraph{Les adultes ne comprennent rien}

"Chrétienne Sauvage n'y comprend rien comme d'habitude". (p107)


Non seulement ils comprennent rien mais aussi ils prennent les enfants pour des idiots.


Notamment au moment de la mort de sa soeur, les adultes pensent que l'enfant ne peut comprendre ("qu'est-ce qu'il peut comprendre" (p192))


\paragraph{Les adultes peuvent se tromper}

\cit "Non, ils n'avaient pas compris; j'étais sûr, comme à l'ordinaire, d'avoir trouvé la faille dans leurs raisonnements"(p109)

\bs

Wole estime que Daodou a une très mauvaise façon de s'y prendre pour corriger les enfants qui chantent mal:

il fouette toute la rangée quand un seul enfant chante faux.

\cit "La vraie solution était évidente, toute simple, mais il ne semblait jamais la considérer" (p335/336)

\bs

Son grand-père le conforte dans ses idées:

"Tu comprendras plus tard. Ils ont essayé de faire une bonne chose, mais ils s'y sont mal pris" (p365)
\bs

Ce qui pêche chez les adultes, ce n'est pas l'intention mais les moyens.


\paragraph{Les adultes sont incohérents et incompréhensibles}


Il y a disqualification de la violence:

Les enfants on interdiction de se battre et s'ils le font, ils sont battus. (incohérent)
\bs



Pour la mère le pire est de faire preuve de l'esprit du diable. (de l'émi esu)



\paragraph{Les adultes sont ridicules}

Le fait de pouvoir rire d'un adulte, c'est pour l'enfant prendre le pouvoir.


\paragraph{Les adultes se comportent mal parfois}

Normalement, les adultes sont du côté du bien.


Certains adultes pourtant, font leurs besoins dans la rue.

Mr Kufogirie, le proviseur adjoint est doté d'une honnêteté moyenne puisque on peut l'acheter pour éviter d'être puni (p315)




\paragraph{Les adultes sont injustes}




\paragraph{La vulnérabilité des adultes}




\section{Enfance et adversité}



\subsection{Les épreuves de la vie}


\subsubsection{La mort et le deuil}




\subsubsection{La misère}





\subsection{La peur, un sentiment récurrent}


\subsubsection{La malveillance des esprits}


\subsubsection{La magie noire}



\subsubsection{Autres peurs}




\subsection{La violence}

\subsubsection{Les adultes sont violents}




\subsubsection{Les corrections physiques}


\paragraph{Eduquer un enfant, c'est le châtier sans faillir}



\paragraph{Or Wole s'interroge sur la légitimité et l'efficacité de ces corrections}


\subsection{La question de l'innocence des enfants}


\subsubsection{La délinquance infantile}


\subsubsection{La cruauté infantile}


\section{Les ressources de l'enfance}


\subsection{La liberté}



\subsection{Le jeu}





\end{document}