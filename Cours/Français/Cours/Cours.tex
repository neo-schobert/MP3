\documentclass[a4paper, 11pt, hidelinks]{article}
\usepackage{bookmark}
\usepackage[utf8]{inputenc} 
\usepackage[T1]{fontenc}
\usepackage{lmodern}
\usepackage{graphicx}
\usepackage[french]{babel}
\usepackage{geometry}
\usepackage{eucal}
\usepackage{caption}
\usepackage{float}
\usepackage{url}
\usepackage{amsmath}
\usepackage{amssymb}
\usepackage{color}
\usepackage{hyperref}
\usepackage{cancel}
\usepackage{romanbar}


\geometry{hmargin=2cm,vmargin=1.5cm}

\newcommand{\dstylesum}{\displaystyle\sum}

\newcommand{\dstyleprod}{\displaystyle\prod}

\newcommand{\prp}{\large \textbf{Proposition :} \large}

\newcommand{\tm}{\large \textbf{Théoreme :} \large}

\newcommand{\ex}{\textcolor{green}{Exemple :} }

\newcommand{\dm}{\textcolor{red}{\textbf{Démo :} } }

\newcommand{\de}{\large \textbf{Définition} \large }

\newcommand{\rmq}{\textbf{Remarque :} }

\newcommand{\bs}{\bigskip}

\newcommand{\voca}{\textcolor{blue}{\textbf{Vocabulaire} } }

\newcommand{\cit}{\large \textcolor{blue}{\textbf{Citation :}} \large }

\newcommand{\rb}[1]{\Romanbar{#1}}
\newcommand{\trinom}[3]{\begin{pmatrix}
    #1 \\
    #2 \\
    #3
\end{pmatrix}}

\newcommand{\quadrinom}[4]{\begin{pmatrix}
    #1 \\
    #2 \\
    #3 \\
    #4 \\
\end{pmatrix}}

\newcommand{\pentanom}[5]{\begin{pmatrix}
    #1 \\
    #2 \\
    #3 \\
    #4 \\
    #5
\end{pmatrix}}

\newcommand{\hexanom}[6]{\begin{pmatrix}
    #1 \\
    #2 \\
    #3 \\
    #4 \\
    #5 \\
    #6 
\end{pmatrix}}

\newcommand{\serie}[2]{\displaystyle\sum_{#1 =0}^{+\infty} #2_{#1} }

\newcommand{\tend}{\underset{n \to + \infty}{\longrightarrow} }

\newcommand{\Lra}{\Leftrightarrow}

\newcommand{\lra}{\leftrightarrow}

\newcommand{\Ra}{\Rightarrow}

\newcommand{\ra}{\rightarrow}

\newcommand{\la}{\leftarrow}

\newcommand{\La}{\Leftarrow}

\newcommand{\dsum}[2]{\displaystyle\sum_{#1}^{#2} }

\newcommand{\dint}[2]{\displaystyle\int_{#1}^{#2} }

\newcommand{\ntend}{\underset{n \to + \infty}{\not \longrightarrow} }

\newenvironment{lmatrix}{$ \left|\begin{array}{l} }{\end{array}\right.$}

\begin{document}




\title{L'enfance dans Ake, les années d'enfance}
\author{Schobert Néo}

\maketitle

\tableofcontents


\newpage




\section{Un récit autobiographique centré autour du personnage de Wole}


\subsection{Un récit autobiographique}


Récit rétrospectif en prose qu'une personne réelle fait de sa propre existence, lorsqu'elle met l'accent sur sa vie individuelle,
en particulier sur l'histoire de sa personnalité.

L'auteur, le narrateur et le personnage sont confondus dans cette première personne qui est le je.


Le roman d'apprentissage est un genre littéraire défini comme l'histoire d'un individu qui va dans le monde pour apprendre à se connaitre
et qui cherche des aventures pour se découvrir et pour donner la pleine mesure de sa personnalité.

Wole rapporte ses souvenirs; mais aussi ceux des autres (en particulier de sa mère).







\subsection{La mémoire}

\subsubsection{Une mémoire sensorielle}



Ce que l'auteur cherche à faire ressentir, c'est "l'invisible réseau de la personnalité d'Aké". (p285)

\bs

\begin{itemize}
    \item On nous parle du "spectacle des odeurs". (p290) (y compris des odeurs désagréables)

Toutes les odeurs se mêlent. (les odeurs de nourriture, d'urine, des femmes)


    \item Au-delà des odeurs, il y aussi les sons qui importent. (les vois)

\cit "Des voix intimes, discrètes, des êtres et des choses qui remplissaient ensemble Aké de l'aube au crépuscule." (p284)


    \item Il y a aussi les couleurs à compter parmi les éléments sensorielle. 

\cit "La poussière la plus rouge qu'il soit possible de trouver sur toute la surface de la Terre." (p250)


    \item Et il y a aussi la nourriture.
\citation "La nourriture du marché." (p294)
\end{itemize}



\subsubsection{Le constat douloureux de l'écoulement du temps et des changements qu'il occasionne}



\cit "Le mystère a été chassé" (p16) citation qui évoque le désenchantement lié à l'écoulement du temps.

Les choses changent certes mais souvent ce changement et vécu comme un désenchantement. Ici, ce qui disparait avec le temps,
c'est la magie (le mystère). C'est souvent très nette avec l'enfance.


\cit "A l'époque dont je parle" (p16). Il utilise souvent le "ne que" qui dévoile le vision pessimiste du changement.

("Ce n'est plus qu'une épave"...)


\cit "Il est arrivé malheur à la mission d'Aké." (p16) Ce désenchantement est même explicite ici.

La voiture a toujours été une épave mais il ne s'en appercoit qu'en étant adulte. Pour les enfants, cette voiture
était un "dragon".


\cit "Ses yeux se sont changés en orbite rouillé, son visage de dragon s'est effondré, ses dents sont peu à peu tombées." (p16)


\bs

Ce désenchantement est subjectif mais aussi objectif. Par exemple, le verger à disparu. Certaines choses deviennent terne car elles sont
moins bien entretenues. On magnifie le passé en général: c'est la nostalgie de l'enfance. Cela est le propre du passage
à l'âge adulte. Vieillir c'est adapté une nouvelle perspective de réalité. Cette nouvelle perspective est marquée par la perte de l'extraordinaire.


\cit "Les perspectives élargies d'une enfance disparu." (p)

\cit "Même la distance qui sépare le campanile du clôcher a raccourci."


Grossomodo, rient n'a changé; ce sont les éléments intimes qui changent:


\cit "Rien de tout cela n'a changé" En disant ça, il ne se contredit pas, il parle globalement


\cit "Mais ce n'est pas le cas des choses plus intimes"







\subsection{Wole: Ecrivain, narrateur et personnage}
\subsubsection{Description de Wole}


\paragraph{a) Un enfant curieux et "raisonneur" (\rb{13}, p354)}


\begin{itemize}
    \item La curiosité
    
    Il est curieux de ce qui l'entoure aussi pour mieux le maîtriser.
    
    On retrouve aussi cela (p314) quand au lycée d'Abeokuta il tente d'éviter le bizutage.

    \cit "Je me mis instinctivement à étudier de très près mes nouveaux compagnons et à rechercher les voies et moyens de survivre au milieu d'eux" (\rb{12}, p314)
    \item Les questions
    Cela amène Wole à poser beaucoup de questions: et cela agace les adultes.

    \cit "On dit que [...] c'est une autre q" (p263 et 276)

    Ce sens du débat on dit qu'il le tient de sons père ce qui énèrve sa mère qui le trouve "trop raisonneur" (p110 et 113). On l'appelle "Mr l'avocat".

    \item Les livres
    Wole va puiser sa curisoté dès qu'il peut dans les livres "Les livres sont un trésor." (p158)
\end{itemize}

\paragraph{b) Un enfant rêveur: "il a toujours été porté à rêvasser" (\rb{7}, p197)}


Pour chrétienne sauvage, la rêverie est une maladie. Elle est inquiète quant à la capacité de l'enfant à s'extraire de ses rêves.
Elle appelle ça la rumination.

Quand Wole déconnecte de la réalité, ça se traduit par des accès de violences. (Voir le massacre des roses ou du petit frère.)
A se moment d'ailleurs, Chrétienne sauvage le défend en disant que Wole était ailleurs: il rêvait.


Chrétienne sauvage met en garde contre les dangers de la rêverie:
\cit "Il faut absolument faire quelque chose avant que tu ne te tues ou que tu mettes feu à la maison" (\rb{5}, p 152)


\cit "Ce trou où je ne me souvenait pratiquement plus de rien" 

\paragraph{c) Un enfant turbulent}



Il fait preuve de mauvaise volonté (d'èmi èsù). Au yeux de la mère, c'est un crime impardonnable.


\subsubsection{La famille de Wole}


\cit "Un besoin d'être avec la famille, de partager l'intimité tranquille du toucher, des regards, dans un rapprochement palpable en chacun de nos actes." (\rb{11}, p308)


\paragraph{a) Le père}


La maison de ses parents est un endroit protecteur pour Aké.
\cit "Pleine de gâteries imprévues" (\rb{5}, p135)

Ces lieux sont des lieux très protecteurs. Il y là bas, "un amour, une protection [...] plus solide, plus vieille que la terre." (\rb{5}, p135)
"A Isara c'était un sentiment constant, indiscutable, et que rien ne pouvait menacer d'ébranler." (C'est ici un sentiment de protection) (\rb{9}, p267)

\cit "Quant à mon père, évidemment, je le tenais pratiquement comme invulnérable" (\rb{1}, p35)


Son père est raffiné; c'est un intelectuel qui aime débattre. Cet aura d'intelectuel lui est donné par son poste de directeur de l'école. (HM: Head master)


Dans ses joutes orales, il va même jusqu'à débattre autour de la religion

\cit "Mon père avait l'habitude de parler comme s'il était à tu et à toi avec Dieu." (\rb{1}, p48)

Cela le rend intransigeant et redoutable "personne n'échappait à Essay." (p147)

A force d'opiniâtreté et d'intransigeance il devient "un personnage étrange qui poursuivait des exigences impossibles avec une telle opiniâtreté." (p142).





\paragraph{b) La mère}



Les enfants ont de nombreuses mères. Parmi toutes ces mères, une sort du lot et semble aimante, il s'agit de la femme du Libraire. 
Elle s'oppose généralement à ce que subissent les enfants.



Les femmes récupèrent les enfants comme s'ils étaient leurs.

Chrétienne sauvage aussi receuille un certain nombre d'enfants.

\cit "C'était elle et non les parents, qui semblait avoir donné naissance à cette famille, l'avoir vomie puis résorbée pour en faire des enfants de la maison" \rb{6} p157


Même les serviteurs de la maison l'appelle maman. C'est la mère de la maison; pas seulement des enfants.

De même, les mots pères et mères peuvent signifier grands-parents. 


Logiquement, si des mères récupèrents des enfants, d'autres démissionnent.


Chrétienen sauvage est un oximore pour plusieurs raisons: 
\begin{itemize}
    \item Société / sauvage
    \item Amour chrétien / sauvage
\end{itemize}


Wole découvre à la fin que chrétienne est opposée à la violence.
\cit "Une incroyable découverte: chrétienne sauvage abhorrait la violence! c'était une révélation stupéfiante." (\rb{14}, p403)




\paragraph{c) La fratrie: "le monde}


Il a:

\begin{itemize}
    \item une s\oe ur qui s'appelle Tinu.
    \item Il a un frère cadet qui s'appelle Ladipo qui va changer de nom car il est trop casse-cou Femi.
    \item Sa s\oe ur Folasade.
\end{itemize}


\cit "Tinu était ma plus proche camarade de jeu, et un lien de protection mutuel s'était établi entre nous, lien qui ne se manifestait jamais que 
lorsqu'elle était blessée ou menacée" (\rb{6}, p159)


A cette fratrie s'ajoute tous les petits abandonnés qui rejoignent la maison.






\section{Le temps de l'enfance}




























































































































\end{document}