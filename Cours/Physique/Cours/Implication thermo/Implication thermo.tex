\documentclass[a4paper, 11pt, hidelinks]{article}
\usepackage{bookmark}
\usepackage[utf8]{inputenc} 
\usepackage[T1]{fontenc}
\usepackage{lmodern}
\usepackage{graphicx}
\usepackage[french]{babel}
\usepackage{geometry}
\usepackage{eucal}
\usepackage{caption}
\usepackage{float}
\usepackage{url}
\usepackage{amsmath}
\usepackage{amssymb}
\usepackage{color}
\usepackage{hyperref}
\usepackage{cancel}
\usepackage{tikz}
\usepackage{mathrsfs}  
\usepackage{esvect}
\usepackage[standard]{ntheorem}


\geometry{hmargin=2cm,vmargin=1.5cm}

\tikzset{
  treenode/.style = {shape=rectangle, rounded corners,
                     draw, align=center,
                     top color=white, bottom color=blue!5},
  root/.style     = {treenode, font=\Large, bottom color=red!10},
  env/.style      = {treenode, font=\ttfamily\normalsize},
  dummy/.style    = {circle,draw}
}

\newcommand{\prp}{\large \textbf{Proposition :} \large}

\newcommand{\tm}{\large \textbf{Théoreme :} \large}

\newcommand{\ex}{\textcolor{green}{Exemple :} }

\newcommand{\dm}{\textcolor{red}{\textbf{Démo :} } }

\newcommand{\de}{\large \textbf{Définition} \large }

\newcommand{\rmq}{\textbf{Remarque :} }

\newcommand{\bs}{\bigskip}

\newcommand{\voca}{\textcolor{blue}{\textbf{Vocabulaire} } }

\newcommand{\lem}{\textcolor{red}{\textbf{Lemme :} } }

\newcommand{\trinom}[3]{\begin{pmatrix}
    #1 \\
    #2 \\
    #3
\end{pmatrix}}

\newcommand{\quadrinom}[4]{\begin{pmatrix}
    #1 \\
    #2 \\
    #3 \\
    #4 \\
\end{pmatrix}}

\newcommand{\pentanom}[5]{\begin{pmatrix}
    #1 \\
    #2 \\
    #3 \\
    #4 \\
    #5
\end{pmatrix}}

\newcommand{\hexanom}[6]{\begin{pmatrix}
    #1 \\
    #2 \\
    #3 \\
    #4 \\
    #5 \\
    #6 
\end{pmatrix}}

\newcommand{\serie}[2]{\displaystyle\sum_{#1 =0}^{+\infty} #2_{#1} }

\newcommand{\tend}{\underset{n \to + \infty}{\longrightarrow} }

\newcommand{\Lra}{\Leftrightarrow}

\newcommand{\lra}{\leftrightarrow}

\newcommand{\Ra}{\Rightarrow}

\newcommand{\ra}{\rightarrow}

\newcommand{\la}{\leftarrow}

\newcommand{\La}{\Leftarrow}

\newcommand{\dsum}[2]{\displaystyle\sum_{#1}^{#2} }

\newcommand{\dint}[2]{\displaystyle\int_{#1}^{#2} }

\newcommand{\ntend}{\underset{n \to + \infty}{\not \longrightarrow} }

\newenvironment{lmatrix}{$ \left|\begin{array}{l} }{\end{array}\right.$}

\newcommand{\img}[4]{\begin{figure}[!ht]
    \centering
    \includegraphics[scale=#1 ]{#2}
    \caption{#3}
    \label{#4}
    \end{figure} }    
\begin{document}

\newcommand{\grad}[1]{\vv{grad}#1}


\title{Processus thermodynamiques}
\author{Schobert Néo}

\maketitle

\tableofcontents


\newpage 


\section{Définition des processus}

\begin{definition}
    Transformation Isotherme : Transformation à température quelconque. 
\end{definition}



\begin{definition}
    Transformation Monotherme : Transformation dans laquelle la température finale est égale à la température initiale.
\end{definition}



\begin{definition}
    Transformation Quasi-statique : Transformation lente dans laquelle les variables d'état
\end{definition}



\begin{definition}
    Transformation Réversible : Transformation qui peut se faire dans un sens comme dans l'autre. ($S_{cree}=0$)
\end{definition}



\begin{definition}
    Transformation Irréversible : Transformation qui ne peut se faire que dans un sens. ($S_{cree} > 0$)
\end{definition}



\begin{definition}
    Transformation Isobare : Transformation à pression constante.
\end{definition}



\begin{definition}
    Transformation monobare : Transformation dans laquelle la pression à l'état final est égale à celle à l'état initial. 
\end{definition}



\begin{definition}
    Transformation Adiabatique : Transformation sans transfert de chaleur
\end{definition}



\begin{definition}
    Transformation Polytropique : Transformation durant laquelle la pression $P$ et le volume $V$ du gaz considéré est de la forme : $PV^m=conste$
\end{definition}


\begin{definition}
    Transformation Isentropique : Transformation à entropie constante. 
\end{definition}


\begin{definition}
    Transformation Isenthalpique : Transformation à enthalpie constante.
\end{definition}




\section{Expression du travail dans différents cas (après application de premier principe)}

\img{0.5}{Travail dans differents cas.png}{Travail dans differents cas}{Figure 1}



\img{0.5}{Travail dans le cas adiabatique reversible.png}{Travail dans le cas adiabatique reversible}{Figure 2}

\section{Loi de Laplace}

Pour une transformation adiabatique réversible, $PV^{\gamma}=constante$


\section{Lien entre chacun des processus}

\begin{itemize}
    \item Réversible $\Rightarrow$ Quasi-statique.
    \item Isotherme $\Rightarrow$ Quasi-statique et Réversible.
    \item Isentropique $\land$ Réversible $\Rightarrow$ Adiabatique.
    \item Isentropique $\land$ Irréversible n'est pas Adiabatique.
    \item Adiabatique $\land$ Réversible $\Rightarrow$ Isentropique.
    \item Adiabatique $\land$ Irréversible n'est pas Isentropique.
    \item \textcolor{red}{Une transformation n'est à la fois adiabatique et isentropique que si elle est réversible.}
\end{itemize}





\end{document}